\documentclass{article}
\usepackage{graphicx}
\usepackage{wrapfig}
\usepackage{filecontents}
\usepackage{siunitx}
\usepackage[table]{xcolor}
\usepackage{float}
\usepackage{hyperref}
\usepackage{color} % balíček pro obarvování textů
\usepackage{xcolor}  % zapne možnost používání barev, mj. pro \definecolor
\hypersetup{
    colorlinks,
    linkcolor={red!50!black},
    citecolor={green!50!black},
    urlcolor={blue!80!black}
}

\usepackage[total={175mm,230mm}, top=23mm, left=20mm, includefoot]{geometry}
\usepackage{pgfplots}
\usepackage{blindtext}

\usepackage{subfiles} % Best loaded last in the preamble

\usepackage{bookmark}

\begin{document}
\small
\section{Úkol}
\begin{enumerate}
    \item Stanovte Planckovu konstantu z měření vnějšího fotoelektrického jevu.
    \item Určete výstupní práci použité fotonky.
\end{enumerate}

\section{Teoretický úvod}
Světlo se může chovat jako vlna ale také jako částice.
Jeho vlnovou formu můžeme popsat jeho frekvencí \(f\) nebo vlnovou délkou \(\lambda\), přičemž platí:
\begin{equation}
    f=\frac{c}{\lambda}
    \label{frek}
\end{equation}
Světlo ve formě částice pak charakterizujeme jeho energií \(E\) a hybností \(p\) a můžeme ho dát do vztahu s jeho vlnovou formou.
\begin{equation}
    E=hf
    \label{ener}
\end{equation}
\begin{equation}
    p=h\frac{f}{c}=h\frac{1}{\lambda}
    \label{hibn}
\end{equation}
Kde \(h\) je Planckova konstanta s hodnotou \(6.626\cdot10^{-34} J\cdot s\) případně \(4.136\cdot10{-15} eV\cdot s\).
Používá se také ve tvaru \(\hbar=\frac{h}{2\pi}\), vztah \ref{ener} pak můžeme přepsat na
\begin{equation}
    E=\hbar\cdot2\pi\cdot f=\hbar\cdot\omega
    \label{E=2p}
\end{equation}
\subsection{Fotoelektrický jev}
Fotoelektrický jev je jev při kterém fotony předávají svou energii elektronům v látce, na kterou dopadá.
Je to jeden z důkazů kvantové povahy světla a můžeme ho rozlišit na vnitřní a vnější.

Vnitřní je typický např. pro polovodiče, ve kterých může foton při dopadu vygenerovat vodivostní pár a tak zvýšit vodivost polovodiče.

Vnější fotoelektrický jev je zase typický pro kovy, ve kterých se díky kovové vazbě mohou elektrony volně pohybovat.
Pokud dopadající foton dodá elektronu alespoň tzv. výstupní práci \(W\), elektron může opustit materiál a pohybovat se volně prostorem rychlostí, která odpovídá energii fotonu bez výstupní práce.
Platí tedy

\begin{equation}
    E_k=hf-W=\frac{hc}{\lambda}
    \label{E_ki}
\end{equation}
Protože elektronu dodává energii foton, jehož energie odpovídá podle vztahu \ref{ener} jeho vlnové délce, je rychlost uvolněného elektronu přímoúměrná vlnové délce dopadajícího světla, zárověň je ale nezávislá na jeho intenzitě.
Každý materiál má tedy jistou mezní fekvenci \(f_m\) pod kterou fotoelektrický jev nenastane.
Na intenzitě dopadajícího záření je ale přímoúměrně závislí počet takto uvolněných elektronů (fotoelektronů).
Tento jev můžeme tedy využít pro konstrukci světelného senzoru, pokud k elektrodě (katoda), na kterou dopadá záření, přiblížíme druhou elektrodu (anoda) uvolněné fotoelektrony budou dopadat na ni.
Bude se tak tvořit napětí, které další fotoelektrony musejí překonávat a tak se fotoelektrony s nižší energií na katodu nedostanou.
Po chvíli se tento jev ustálí a vznikne tak brzdné napětí \(U_b\).
Fotoelektrony musejí krom vystupní práce mít ještě dost energie na překonání bariéry vytvořené brzdným napětím, tato energie odpovídá \(qU_b\) a energie fotonu musí tedy odpovídat 
\begin{equation}
    qU_b=E_k=hf-W=\frac{hc}{\lambda}
    \label{eU_b}
\end{equation}
Kde \(q\) je elementární náboj elektronu.
Můžeme tedy stanovit závislost \(U_b\) na frekvenci respektive vlnové délce dopadajícího záření.
\begin{equation}
    U_b=\frac{hf}{q}-\frac{W}{q}=\frac{h}{q}f-\frac{h}{q}f_m=\frac{h}{q}(f-f_m)
    \label{U_b-}
\end{equation}
Planckovu konstantu můžeme úpravou vztahu \ref{U_b-} vyjádřit jako
\begin{equation}
    h=\frac{U_mq}{-f_m}
    \label{plan}
\end{equation}
Mezní frekvenci můžene určit úpravou vztahu
\begin{equation}
    U_b=\frac{h}{q}(f-f_m)=0 ~~ \Longrightarrow ~~ f_m=f-\frac{U_bq}{h}
    \label{f_m}
\end{equation}
Nakonec výstupní práci vyjádříme ze vztahu \ref{U_b-} při \(f=0\)
\begin{equation}
    U_{b0}=\frac{hf}{q}-\frac{W}{q}=\frac{h0}{q}-\frac{W}{q}=-\frac{W}{q}\Rightarrow W=-gU_{b0}
    \label{V-b0}
\end{equation}

\section{Měření}
\begin{figure}[H]
	\begin{minipage}[t]{0.5\textwidth}
        \begin{tabular}{|c|c|c|c|c|}
            \hline
            \(\lambda~[nm]\)    & \(U_{vystup}~[V]\)    & zesílení  & \(f~[THz]\)   & \(U_b~[V]\)   \\ \hline
            366                 & 1.915                 & \(10^0\)  & 819.11        & 1.84          \\ \hline
            405                 & 1.726                 & \(10^0\)  & 740.23        & 1.65          \\ \hline
            436                 & 1.491                 & \(10^0\)  & 687.60        & 1.42          \\ \hline
            546                 & 9.15                  & \(10^1\)  & 549.07        & 0.84          \\ \hline
            578                 & 8.19                  & \(10^1\)  & 518.67        & 0.74          \\ \hline
        \end{tabular}
	\end{minipage}
	\hfill
	\begin{minipage}[t]{0.5\textwidth}
        \footnotesize
        \vspace{-17mm}
        \subsection*{Použité měřící přístroje}
        \begin{tabular}{llll}
            Multimetr               & 703   & FINEST    & SAP\:001000282754\_0000 \\ \hline
            Měřící zesilovač        &       & PHYWE     & 313384                  \\ \hline
            Speciální žárovka       &       &           &                         \\ \hline
            Přípravek s fotonkou    &       &           &                         \\ \hline
            Sada optických filtrů   &       &           &                         \\ \hline    
        \end{tabular}
	\end{minipage}
\end{figure}

\begin{figure}[H]
	\begin{minipage}[t]{0.5\textwidth}
        \hspace{-10mm}
        \pgfplotsset{width=120mm,compat=1.9}
        \tikzset{
        every pin/.style={fill=yellow!50!white,rectangle,rounded corners=3pt,font=\scriptsize},
        small dot/.style={fill=black,circle,scale=0.3}
        % /dotted
        }
        \begin{tikzpicture}
            \begin{axis}[
                % colorbar,
                title={Graf závislosti \(U_b\) na frekvenci \(f\)},
                xlabel={\(f~[THz]\)},
                ylabel={\(U_b~[V]\)},
                xmin=0, xmax=850,
                ymin=-1.8, ymax=2,
                legend pos=north west,
            ]
            \addplot[
                only marks,
                color=red,
                mark=*,
                ]
                coordinates {
                    (819.11, 1.84)
                    (740.23, 1.65)
                    (687.60, 1.42)
                    (549.07, 0.84)
                    (518.67, 0.74)
                    };
                \addlegendentry{měření}
            \addplot[
                % tick style={font=/dotted}, %/tikz
                domain=0:900, 
                samples=500,
                color=green,
                dashed, %dotted
                ]
                {(x/242)-1.43};
                \addlegendentry{extrapolace (\(\frac{1}{242}f-1.43\))}
            \addplot[
                % tick style={font=/dotted}, %/tikz
                domain=0:100, 
                samples=2,
                dotted
                ]
                coordinates {
                    (      0, 0)
                    ( 346.06, 0)
                    ( 346.06,-2)
                    };
                % \addlegendentry{\(\lambda=470~nm\)}
            \node[small dot,pin=0:{\texttt{\([0~THz;-1.43~V]\)}}] at (axis description cs:0,0.09736842105263162) {};
            \node[small dot,pin=93:{\texttt{\([346.06~THz;0~V]\)}}] at (axis description cs:0.4071294117647059,0.4736842105263158) {};
            \end{axis}
        \end{tikzpicture}
    \end{minipage}
	\hfill
    \hspace{30mm}
	\begin{minipage}[t]{0.35\textwidth}
        \vspace{-96mm}
        ze vztahu \ref{plan} určíme Planckovu konstantu \(h\) \\
        \\
        \(
            \scriptsize
            h=\frac{U_mq}{-f_m}=\frac{-1.43\cdot1.602\cdot10^{-19}}{-346.06\cdot10^{12}}~[J\cdot s]=\\\\=6.619\cdot10^{-34}~[J\cdot s]
        \)\\
        \\
        % ze vstahu \ref{f_m} dále určíme mezní frekvenci \(f_m\)\\
        % \\
        % \(
        %     \scriptsize
        %     f_m=f-\frac{U_bq}{h}=\\\\=\big(549.07\cdot10^{12}-\frac{0.84\cdot1.602\cdot10^{-19}}{6.619\cdot10^{-34}}\big)~[THz]=\\\\=345.979~[THz]
        % \)\\
        % \\
        nakonec podle vztahu \ref{V-b0} určíme výstupní práci\\
        \\
        \(
            \scriptsize
            W=-gU_{b0}=\\\\=-1.602\cdot10^{-19}\cdot(-1.43)~[J]=\\\\=2.29\cdot10^{-19}~[J]=1.43~[eV]
        \)\\
        \\
    \end{minipage}
\end{figure}

\section{Závěr}
Pět měření jsem vynesl do grafu a na základě takto vzniklých bodů jsem aproximoval závislost brzdného napětí \(U_b\) na frekvenci dopadajícího záření \(f\).
Pomocí aproximace jsem stanovil brzdné napětí \(U_b\) při nulové frekvenci \(U_{b0}\) a mezní frekvenci \(f_m\).
Z takto získaných hodnot jsem následně určil Planckovu konstantu \(h = 6.619\cdot10^{-34}~[J\cdot s]\) a výstupní práci použité fotonky \(W = 2.29\cdot10^{-19}~[J]\) neboli \(W =1.43~[eV]\).
Tabulková hodnota Planckovy konstanty je \(6.626\cdot10^{-34}~[J\cdot s]\) odchylka mého měření je tedy \(7\cdot10^{-37}~[J\cdot s]\), což nezní špatně.

\end{document}
