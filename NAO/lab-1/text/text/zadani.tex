Detailní popis jednotlivých úloh s návodem najdete v NAO\_PC.pdf, který je dostupný v E-learningu.

\begin{itemize}
    \item {\bf Simulací získejte hodnoty prahového napětí \(U_{TH0}\) pro dvě různé řady rozměrů tranzistorů.}
    \begin{itemize}
        \item konstantní poměr \(W/L = 5, kdy \$L\$ = 0.18; 0.3; 0.5; 0.8; 1; 2; 3; 5; 10\),
        \item různé rozměry: \(W/L = 0.22/0.18; 1/0.5; 2/0.5; 2/1; 5/1; 5/2; 10/5; 10/10; 40/10\),
        \item výše uvedené dva body budou provedeny pro tranzistor NMOS i PMOS.
    \end{itemize}
    \item {\bf Závislost prahového napětí \(U_{TH}\) na \(U_{SB}/U_{BS}\) (bulk efekt) Simulací získejte hodnoty prahového napětí \(U_{TH}\) pro napětí \(U_{BS}\) (NMOS) resp. \(U_{SB}\) (PMOS) v rozsahu \(0 [V]\) až \(500 [mV]\) s krokem \(50 [mV]\).}
    \item {\bf Závislost parametru modulace délky kanálu (\(\lambda\)) na délce kanálu (L) Simulací získejte hodnoty parametru \(\lambda\) tranzistoru NMOS a PMOS pro L v rozmezí \(200 [nm]\) až \(10 [\mu m]\).}
\end{itemize}

\subsection*{Bonusové otázky (1 b.)}
\begin{itemize}
    \item Popište, jak byste simulací (mimo analýzu \inlinecode{.op}) zjistili transkonduktanci \(gm\).
          Nakreslete schéma, popište nastavení a odečet hodnot.
\end{itemize}
