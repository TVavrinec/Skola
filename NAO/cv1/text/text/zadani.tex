Simulacemi zjistěte tyto parametry tranzistorů NMOS a PMOS:

\begin{enumerate}
    \item {\bf Transkonduktační parametr \(KP\)}
    \begin{itemize}
        \item Při ID = 10 µA
    \end{itemize}
    \item {\bf Prahového napětí UTH0 pro dvě různé řady rozměrů tranzistorů.}
    \begin{enumerate}
        \item konstantní poměr \(W/L = 5\), kdy 
            \(L = 0.18u, 0.3u, 0.5u, 0.8u, 1u, 2u, 3u, 5u, 10u\); potom:
            \(W = 0.9u, 1.5u, 2.5u, 4u, 5u, 10u, 15u, 25u, 50u\).
        \item různé rozměry: \(W/L = 0.22u/0.18u; 1u/0.5u; 2u/0.5u; 2u/1u; 5u/1u; 5u/2u; 10u/5u; 10u/10u; 40u/10u,\)
    \end{enumerate}
    \item {\bf Závislost prahového napětí \(UTH\) na \(USB/UBS\) (bulk efekt)}
    \begin{itemize}
        \item Simulací získejte hodnoty prahového napětí UTH pro napětí UBS (NMOS) resp. USB (PMOS) v rozsahu \(0 V\) až \(1 V\) s krokem \(100 mV.\) \(W/L = (5/1) \mu m\)
    \end{itemize}
    \item Závislost parametru modulace délky kanálu (\(\lambda\)) na délce kanálu (\(L\))
    \begin{itemize}
        \item Simulací získejte hodnoty parametru \(\lambda\) pro \(L\) v rozmezí \(200 nm\) až \(10 \mu m\). \(W/L = 5\).
    \end{itemize}
\end{enumerate}

Výstupem do elearningu bude soubor pdf s přehledně zpracovatelnými parametry v tabulkách. 
