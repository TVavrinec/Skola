\documentclass{protokol}
\usepackage{graphicx}
\usepackage{float}

\usepackage{tikz}
\usetikzlibrary{calc}
\usetikzlibrary{arrows}


%====== Units =====
\usepackage{amssymb}
\usepackage{siunitx}
\sisetup{inter-unit-product =\ensuremath{\cdot}}
\sisetup{group-digits = integer}
\sisetup{output-decimal-marker = {,}}
\sisetup{exponent-product = \ensuremath{\cdot}}
\sisetup{separate-uncertainty}
\sisetup{tight-spacing = false}
%\sisetup{scientific-notation = true}
%\sisetup{round-mode=places,round-precision=4}
%\sisetup{evaluate-expression}


%====== Grafy =====
\usepackage{pgfplots}
\pgfplotsset{width=0.8\linewidth, compat=1.17}
\def\plotcscale{0.8}
\usepackage{pgfplotstable}
\usepackage[figurename=Obr.]{caption} % figure caption rename

\usepackage[siunitx]{circuitikz} % obvody
% \ctikzset{bipoles/length=1cm}

%====== Rovnice align block ======
\usepackage{amsmath}
\setlength{\jot}{10pt} % rozestup mezi řádky

\graphicspath{ {./img/} }

%====== Code blocks ======
\usepackage{listings}

\usepackage[most]{tcolorbox}
\newtcbox{\inlinecode}{on line, boxsep=0pt, left=2pt, right=2pt, top=2pt, bottom=2pt, colframe=gray, boxrule=0.5pt, arc=1pt, auto outer arc, before upper={\vphantom{dlg}}, tcbox raise base} % vytvoření inline boxu pro kód

\lstdefinelanguage{Spice}{
    alsoletter={.},  % Přidává tečku jako součást slov
    morekeywords={.lib, .STEP, .param, .MEAS, FIND, WHEN, END, VALUE, R, C, L, K, .SUBCKT, .ENDS, .MODEL, .INCLUDE, .AC, .DC, .TRAN, .PRINT, .PLOT, .PROBE, .WIDTH, .OPTIONS, .NODESET, .IC},
    keywordstyle=\color{blue}\bfseries,
    sensitive=false, % klíčová slova nejsou citlivá na velikost písmen
    morecomment=[l]{;}, % komentáře začínají hvězdičkou
    morestring=[b]", % řetězce uzavřené do uvozovek
}

\lstset{
  language=Spice,
  basicstyle=\ttfamily\small,
  breaklines=true,
  numbers=left,
  numberstyle=\tiny\color{gray},
  frame=single,
  rulecolor=\color{black},
  title=\lstname,
  keywordstyle=\color{blue}\bfseries,
  commentstyle=\color{green},
  stringstyle=\color{red},
  showstringspaces=false
}

%====== Vyplňte údaje ======
\jmeno{Tomáš Vavrinec}
\kod{240893}
\rocnik{}
\obor{MET}
\skupina{}
\spolupracoval{--}

\merenodne{--}
\odevzdanodne{--}
\nazev{1.	Proudová zrcadla}
\cislo{2} %měřené úlohy

\predmet{Návrh analogových integrovaných obvodů}
\ustav{Ústav mikroelektroniky}
\skola{FEKT VUT v~Brně}

\def\para{x+0}
\def\parb{\para-80}


%citace 
\usepackage[backend=biber, style=iso-numeric, sortlocale=cs_CZ, autolang=other, language=czech]{biblatex}
\addbibresource{bibliography.bib}
\DeclareFieldFormat{labelnumberwidth}{\mkbibbrackets{#1}}
% hyperlinky
\usepackage[colorlinks]{hyperref}

% odstavce
\usepackage{parskip}

% Bloky kódu
\usepackage{xcolor}

%New colors defined below
\definecolor{codegreen}{rgb}{0,0.6,0}
\definecolor{codegray}{rgb}{0.5,0.5,0.5}
\definecolor{codepurple}{rgb}{0.58,0,0.82}
\definecolor{backcolour}{rgb}{0.95,0.95,0.92}

\usepackage{listings}
\lstdefinestyle{mystyle}{
  backgroundcolor=\color{backcolour}, commentstyle=\color{codegreen},
  keywordstyle=\color{magenta},
  numberstyle=\tiny\color{codegray},
  stringstyle=\color{codepurple},
  basicstyle=\ttfamily\footnotesize,
  breakatwhitespace=false,         
  breaklines=true,                 
  captionpos=b,                    
  keepspaces=true,                 
  numbers=left,                    
  numbersep=5pt,                  
  showspaces=false,                
  showstringspaces=false,
  showtabs=false,                  
  tabsize=2
}
\lstset{
	inputencoding=utf8,
	extendedchars=true,
	literate={á}{{\'a}}1 {č}{{\v{c}}}1 {ď}{{\v{d}}}1 {é}{{\'e}}1 {ě}{{\v{e}}}1 
           {í}{{\'i}}1 {ň}{{\v{n}}}1 {ó}{{\'o}}1 {ř}{{\v{r}}}1 {š}{{\v{s}}}1 
           {ť}{{\v{t}}}1 {ú}{{\'u}}1 {ů}{{\r{u}}}1 {ý}{{\'y}}1 {ž}{{\v{z}}}1 
           {Á}{{\'A}}1 {Č}{{\v{C}}}1 {Ď}{{\v{D}}}1 {É}{{\'E}}1 {Ě}{{\v{E}}}1 
           {Í}{{\'I}}1 {Ň}{{\v{N}}}1 {Ó}{{\'O}}1 {Ř}{{\v{R}}}1 {Š}{{\v{S}}}1 
           {Ť}{{\v{T}}}1 {Ú}{{\'U}}1 {Ů}{{\r{U}}}1 {Ý}{{\'Y}}1 {Ž}{{\v{Z}}}1,
	style=mystyle
	}

% Číslování
\pagenumbering{arabic}



% =========================================
% =============== DOKUMENT ================
% =========================================
\begin{document}
	%====== Vygenerování tabulky ======
	\maketitle

\section*{ZADÁNÍ ÚLOHY}
  Simulacemi zjistěte tyto parametry tranzistorů NMOS a PMOS:

\begin{enumerate}
    \item {\bf Transkonduktační parametr \(KP\)}
    \begin{itemize}
        \item Při ID = 10 µA
    \end{itemize}
    \item {\bf Prahového napětí UTH0 pro dvě různé řady rozměrů tranzistorů.}
    \begin{enumerate}
        \item konstantní poměr \(W/L = 5\), kdy 
            \(L = 0.18u, 0.3u, 0.5u, 0.8u, 1u, 2u, 3u, 5u, 10u\); potom:
            \(W = 0.9u, 1.5u, 2.5u, 4u, 5u, 10u, 15u, 25u, 50u\).
        \item různé rozměry: \(W/L = 0.22u/0.18u; 1u/0.5u; 2u/0.5u; 2u/1u; 5u/1u; 5u/2u; 10u/5u; 10u/10u; 40u/10u,\)
    \end{enumerate}
    \item {\bf Závislost prahového napětí \(UTH\) na \(USB/UBS\) (bulk efekt)}
    \begin{itemize}
        \item Simulací získejte hodnoty prahového napětí UTH pro napětí UBS (NMOS) resp. USB (PMOS) v rozsahu \(0 V\) až \(1 V\) s krokem \(100 mV.\) \(W/L = (5/1) \mu m\)
    \end{itemize}
    \item Závislost parametru modulace délky kanálu (\(\lambda\)) na délce kanálu (\(L\))
    \begin{itemize}
        \item Simulací získejte hodnoty parametru \(\lambda\) pro \(L\) v rozmezí \(200 nm\) až \(10 \mu m\). \(W/L = 5\).
    \end{itemize}
\end{enumerate}

Výstupem do elearningu bude soubor pdf s přehledně zpracovatelnými parametry v tabulkách. 

\setcounter{section}{0}

\newpage
\section{Vypracování}
  \subsection{Kaskodové proudové zrcadlo} 
  Jako první určíme rozměry tranzistoru zrcadla.
Volím délku kanálu \(L = 2 [\mu m]\) jako kompromis mezi velikostí a parametrem \(\lambda\) která pro \(L = 2 \mu m\) nabívá hodnoty \(\lambda = 0.0787698 [V^{-1}]\).
Dále musíme zvolit napětí \(U_{OV}\) které volím s ohledem na rozsah napájecího napětí \(U_{OV} = 0.2 [V]\)
Z toho následně můžeme určit šířku kanálu \(W\), tranzistorů \(M2\) a \(M3\) jako:

\begin{center}
    \large
    \(
        W_{M2} = W_{M3} = L \cdot \frac{2 \cdot I_1}{KP \cdot U_{OV}^2} = 2\mu \cdot \frac{2 \cdot 10\mu}{50\mu 0.2^2} = 20 [\mu m]
    \)
\end{center}

Z Toho snadno určíme \(W_{M4}\) a \(W_{5}\) jako:

\begin{center}
    \large
    \(
        W_{M4} = W_{M5} = W_{M2} \cdot \frac{I_2}{I_1} = 20\mu \cdot \frac{20\mu}{10\mu} = 40 [\mu m]
    \)
\end{center}

Výstupní odpor pak můžeme určit jako:

\begin{center}
    \large
    \(
        r_{out} = g_{m-M4} \cdot r_{DS-M4} \cdot r_{DS-M5} = g_{m-M4} \cdot \frac{1}{\lambda_{M4}\cdot I_{M4}} \cdot \frac{1}{\lambda_{M5}\cdot I_{M5}} = 73.2\mu \cdot \frac{1}{0.0787698 \cdot 20\mu} \cdot \frac{1}{0.0787698 \cdot 20\mu} = 29.5 [M \Omega]
    \)
\end{center}

A výstupní rozsah jako:

\begin{center}
    \large
    \(
        U_{out} = U_{CC} - (U_{OV-M5} + U_{OV-M4} + U_{TH-M4}) = 1.8 - (0.2+0.2+0.427) = 0.97 [V]
    \)
\end{center}

\vspace{10mm}
\begin{figure}[h!]
    \centering
    \includegraphics[width=0.9\textwidth]{text/img/KPZ-op-sch.png}
    \caption{\label{fig:KPZ-op-sch} zobrazení napětí a proudu ve schématu}
\end{figure}

\begin{figure}[h!]
    \centering
    \includegraphics[width=0.9\textwidth]{text/img/KPZ-op-OL.png}
    \caption{\label{fig:KPZ-op-OL} Pracovní body jednotlivých tranzistorů}
\end{figure}

\newpage

\begin{figure}[h!]
    \centering
    \includegraphics[width=0.9\textwidth]{text/img/KPZ-dc-graf.png}
    \caption{\label{fig:KPZ-dc-graf} Simulovaná závislost výstupího proudu na výstupním napětí}
\end{figure}

\begin{figure}[h!]
    \centering
    \includegraphics[width=0.9\textwidth]{text/img/KPZ-dc-graf-detail.png}
    \caption{\label{fig:KPZ-dc-graf-detail} Simulovaná závislost výstupího proudu na výstupním napětí v detailu}
\end{figure}

Z grafu \ref{fig:KPZ-dc-graf-detail} můžeme odečíst pokles hodnoty proudu na obou stranách pracovního rozsahu a i přímo jejich změny \(\Delta I = 14.14 [nA]\) při změně napětí \(\Delta U = 970.7 [mV]\).
Výstupní odpor tak můžeme určit jako:

\begin{center}
    \large
    \(
        r_{out} = \frac{\Delta U}{\Delta I} = \frac{970.7m}{14.14n} = 68.2 [M\Omega] 
    \)
\end{center}

Tato hodnota je cca poloviční ve srovnání s výpočtem, původně jsem předpokládal že mám špatné hodnoty \(\lambda\) ale ani po kontrole a opravě jsem nedošel k výsledkům podobným simulaci.

Výstupní rozsah odpovídá naproti tomu výpočtu velmi přesně a na průběhu je i vidět dvě místa kde dochází ke změnám výstupního odporu.
Nejdřív menší změna hned za koncem rozsahu když začne \(M4\) přecházet do lineárního režimu a následně vetší změna když začne vstupovat do lineárního režimu i \(M5\).

  \newpage
  \clearpage

  \subsection{Wilznovo proudové zrcadlo} 
  Jako první určíme rozměry tranzistoru zrcadla.
Volím délku kanálu \(L = 2 [\mu m]\) jako kompromis mezi velikostí a parametrem \(\lambda\), která pro \(L = 2 \mu m\) nabývá hodnoty \(\lambda = 0.0441692 [V^{-1}]\).
Dále musíme zvolit napětí \(U_{OV}\), které volím s ohledem na rozsah napájecího napětí \(U_{OV} = 0.2 [V]\)
Z toho následně můžeme určit šířku kanálu \(W\), tranzistorů \(M1\) jako:

\begin{center}
    \large
    \(
        W_{M1} = L \cdot \frac{2 \cdot I_1}{KP \cdot U_{OV}^2} = 2\mu \cdot \frac{2 \cdot 10\mu}{200\mu 0.2^2} = 5 [\mu m]
    \)
\end{center}

Z toho snadno určíme \(W_{M4}\) a \(W_{5}\) jako:

\begin{center}
    \large
    \(
        W_{M2} = W_{M3} = W_{M1} \cdot \frac{I_2}{I_1} = 5\mu \cdot \frac{20\mu}{10\mu} = 10 [\mu m]
    \)
\end{center}

Při určování hodnoty odporu nastavujícího proud \(I_1\) musíme vzít v potas bulk efekt tranzistoru  a určíme jako:

\begin{center}
    \large
    \(
        R = \frac{V_{CC} - (U_{TH-M2} + U_{OV-M2} + U_{TH-M3} + U_{OV-M3})}{I_1} = \frac{1.8 - (0.384 + 0.2 + 0.541 + 2)}{10\mu} = 47.4 [k\Omega] 
    \)
\end{center}

Výstupní rozsah můžeme určit jako:
\begin{center}
    \large
    \(
        U_{out} = U_{CC} - (U_{TH-M2} + U_{OV-M2} + U_{OV-M3}) = 1.8 - (0.363+0.2+0.2) = 1.037 [V]
    \)
\end{center}

Následně můžeme určit výstupní odpor jako:
\begin{center}
    \Large
    \(
        R_{out} = r_{ds3} \left[ 1 + g_{m1} \cdot \frac{r_{ds1} \cdot R}{r_{ds1} + R} \right] = \frac{1}{\lambda \cdot I_2} \left[ 1 + g_{m1} \cdot \frac{\frac{1}{\lambda \cdot I_1} \cdot R}{\frac{1}{\lambda \cdot I_1} + R} \right]
    \)
    \(
        R_{out} = \frac{1}{0.0441692 \cdot 20\mu} \cdot \left[ 1 +  140.8\mu \cdot \frac{\frac{1}{0.0441692 \cdot 10\mu} \cdot 47.4k}{\frac{1}{0.0441692 \cdot 10\mu} + 47.4k}\right] = 8.532 [M\Omega]
    \)
\end{center}

\begin{figure}[h!]
    \centering
    \includegraphics[width=0.6\textwidth]{text/img/WPZ-op-OL.png}
    \caption{\label{fig:WPZ-op-OL} Pracovní body jednotlivých tranzistorů}
\end{figure}

\begin{figure}[h!]
    \centering
    \includegraphics[width=0.9\textwidth]{text/img/WPZ-op-sch.png}
    \caption{\label{fig:WPZ-op-sch} Výsledné schéma s vyznačenými proudy \(I_1\) a \(I_2\)}
\end{figure}

\begin{figure}[h!]
    \centering
    \includegraphics[width=0.9\textwidth]{text/img/WPZ-dc-graf.png}
    \caption{\label{fig:WPZ-dc-graf} Závislost vstupního a výstupního proudu na výstupním napětí}
\end{figure}

Ze simulace můžeme určit výstupní odpor jako:
\begin{center}
    \large
    \(
        r_{out} = \frac{\Delta U}{\Delta I} = \frac{951.4m}{69.814474n} = 14.285 [M\Omega] 
    \)
\end{center}

Což je téměř dvojnásobek oproti výpočtu, výstupní rozsah naproti tomu sedí poměrně přesně.
  \newpage

% \section{Závěr}
%   Jednotlivé metody lámání měli různé výsledky.
Nejlépe se mi osvědčil diamantoví nuž, který jako jediný vytvořil rovný kolmí řez.

% \clearpage
% \section*{Reference}
% \printbibliography[heading=none]

\end{document}