\documentclass{article}
\usepackage{graphicx}
\usepackage{wrapfig}
\usepackage{filecontents}
\usepackage{siunitx}
\usepackage[table]{xcolor}
\usepackage{float}
\usepackage{hyperref}

\usepackage{color} % balíček pro obarvování textů
\usepackage{xcolor}  % zapne možnost používání barev, mj. pro \definecolor
\usepackage{pgfplots} % http://www.chiark.greenend.org.uk/doc/texlive-doc/latex/pgfplots/pgfplots.pdf

\ifnum 0\ifxetex 1\fi\ifluatex 1\fi=0 % if pdftex
  \usepackage[T1]{fontenc}
  \usepackage[utf8]{inputenc}
\else % if luatex or xelatex
  \ifxetex
    \usepackage{mathspec}
  \else
    \usepackage{fontspec}
  \fi
  \defaultfontfeatures{Ligatures=TeX,Scale=MatchLowercase}
\fi
\usepackage[total={175mm,230mm}, top=23mm, left=20mm, includefoot]{geometry}
\hypersetup{
    colorlinks,
    linkcolor={blue!50!black},
    citecolor={green!50!black},
    urlcolor={blue!80!black}
}
% \definecolor{fialova}{RGB}{ 255, 000, 255}
\definecolor{color-si1}{RGB}{ 255, 000, 000}
\definecolor{color-si2}{RGB}{ 251, 130, 032}

\definecolor{color-ge1}{RGB}{ 000, 255, 000}
\definecolor{color-ge2}{RGB}{ 032, 251, 160}

\definecolor{color-inp1}{RGB}{ 000, 000, 255}
\definecolor{color-inp2}{RGB}{ 160, 032, 251}

\definecolor{color-geas1}{RGB}{ 225, 225, 000}
\definecolor{color-geas2}{RGB}{ 225, 225, 100}

\definecolor{sedak}{RGB}{ 100, 100, 100}


\newcommand \obr[1]
{ obr.~\ref{#1}}

\newcommand \tab[1]
{ tab.~ß\ref{#1}}


\begin{document}

\pagestyle{empty}

\definecolor{color_29791}{rgb}{0,0,0}
\begin{figure}[H]
    \hspace{-13mm}
    \begin{minipage}[t]{\textwidth}
        \vspace{-20mm}
        \begin{tikzpicture}[overlay]
            \path(0pt,0pt);
        \end{tikzpicture}
        \begin{picture}(-5,0)(2.5,0)
            \put(123.656,-82.75397){\fontsize{18}{1}\usefont{T1}{ptm}{m}{n}\selectfont\color{color_29791}VYSOKÉ UČENÍ TECHNICKÉ V BRNĚ}
            \put(76.296,-104.785){\fontsize{13}{1}\usefont{T1}{ptm}{m}{n}\selectfont\color{color_29791}FAKULTA  ELEKTROTECHNIKY A KOMUNIKAČNÍCH TECHNOLOGIÍ}
            \put(198.447,-128.5339){\fontsize{16}{1}\usefont{T1}{cmr}{b}{n}\selectfont\color{color_29791}Ústav elektrotechnologie}
            \put(156.848,-278.1589){\fontsize{14}{1}\usefont{T1}{ptm}{m}{n}\selectfont\color{color_29791}LABORATORNÍ CVIČENÍ Z PŘEDMĚTU}
            \put(108.123,-300.2579){\fontsize{14}{1}\usefont{T1}{cmr}{b}{n}\selectfont\color{color_29791}VYBRANÉ PARTIE Z OBNOVITELNÝCH ZDROJŮ A}
            \put(173.123,-320.2579){\fontsize{14}{1}\usefont{T1}{cmr}{b}{n}\selectfont\color{color_29791}UKLÁDÁNÍ ENERGIE (BPC-OZU)}
            \put(55.85,-421.25){\fontsize{14}{1}\usefont{T1}{cmr}{b}{n}\selectfont\color{color_29791}Číslo úlohy: 7}
            \put(55.85,-469.547){\fontsize{14}{1}\usefont{T1}{cmr}{b}{n}\selectfont\color{color_29791}Název úlohy: Využití termoelektrického jevu pro získávání energie}
            \put(23.9,-620.32){\fontsize{12}{1}\usefont{T1}{cmr}{b}{n}\selectfont\color{color_29791}Jméno a příjmení, ID:}
            \put(23.9,-637.119){\fontsize{12}{1}\usefont{T1}{cmr}{b}{n}\selectfont\color{color_29791}Tomáš Vavrinec, 240893}
            \put(186.95,-620.32){\fontsize{12}{1}\usefont{T1}{cmr}{b}{n}\selectfont\color{color_29791}Atmosférický tlak:}
            \put(186.95,-637.119){\fontsize{12}{1}\usefont{T1}{cmr}{b}{n}\selectfont\color{color_29791}1018 hPa}
            \put(293.25,-620.32){\fontsize{12}{1}\usefont{T1}{cmr}{b}{n}\selectfont\color{color_29791}Teplota okolí: }
            \put(293.25,-637.119){\fontsize{12}{1}\usefont{T1}{cmr}{b}{n}\selectfont\color{color_29791}21.7°C}
            \put(417.25,-620.32){\fontsize{12}{1}\usefont{T1}{cmr}{b}{n}\selectfont\color{color_29791}Relativní vlhkost:}
            \put(417.25,-637.119){\fontsize{12}{1}\usefont{T1}{cmr}{b}{n}\selectfont\color{color_29791}24.6\%}
            \put(23.9,-665.77){\fontsize{12}{1}\usefont{T1}{cmr}{b}{n}\selectfont\color{color_29791}Měřeno dne:}
            \put(23.9,-682.569){\fontsize{12}{1}\usefont{T1}{cmr}{b}{n}\selectfont\color{color_29791}25.2.2023}
            \put(186.95,-665.77){\fontsize{12}{1}\usefont{T1}{cmr}{b}{n}\selectfont\color{color_29791}Odevzdáno dne:}
            \put(293.25,-665.77){\fontsize{12}{1}\usefont{T1}{cmr}{b}{n}\selectfont\color{color_29791}Ročník, stud. skupina:}
            \put(293.25,-682.569){\fontsize{12}{1}\usefont{T1}{cmr}{b}{n}\selectfont\color{color_29791}2}
            \put(417.25,-665.77){\fontsize{12}{1}\usefont{T1}{cmr}{b}{n}\selectfont\color{color_29791}Kontrola:}
            \put(23.9,-703.42){\fontsize{12}{1}\usefont{T1}{cmr}{b}{n}\selectfont\color{color_29791}Spolupracovali:}
            \put(23.9,-720.219){\fontsize{12}{1}\usefont{T1}{cmr}{b}{n}\selectfont\color{color_29791}Kateřina Koudelková}
        \end{picture}
        \begin{tikzpicture}[overlay]
            \path(0pt,0pt);
            \draw[color_29791,line width=0.5pt]
            (20.4pt, -606.117pt) -- (20.4pt, -722.815pt)
            ;
            \draw[color_29791,line width=0.5pt]
            (183.45pt, -606.117pt) -- (183.45pt, -651.067pt)
            ;
            \draw[color_29791,line width=0.5pt]
            (183.45pt, -651.567pt) -- (183.45pt, -688.717pt)
            ;
            \draw[color_29791,line width=0.5pt]
            (289.75pt, -606.117pt) -- (289.75pt, -651.067pt)
            ;
            \draw[color_29791,line width=0.5pt]
            (289.75pt, -651.567pt) -- (289.75pt, -688.717pt)
            ;
            \draw[color_29791,line width=0.5pt]
            (413.75pt, -606.117pt) -- (413.75pt, -651.067pt)
            ;
            \draw[color_29791,line width=0.5pt]
            (413.75pt, -651.567pt) -- (413.75pt, -688.717pt)
            ;
            \draw[color_29791,line width=0.5pt]
            (544.9pt, -606.117pt) -- (544.9pt, -722.815pt)
            ;
            \draw[color_29791,line width=0.5pt]
            (20.15pt, -605.867pt) -- (545.15pt, -605.867pt)
            ;
            \draw[color_29791,line width=0.5pt]
            (20.65pt, -651.317pt) -- (544.65pt, -651.317pt)
            ;
            \draw[color_29791,line width=0.5pt]
            (20.65pt, -688.967pt) -- (544.65pt, -688.967pt)
            ;
            \draw[color_29791,line width=0.5pt]
            (20.15pt, -723.065pt) -- (545.15pt, -723.065pt)
            ;
            \draw[color_29791,line width=1.5pt]
            (15.75pt, -15.59998pt) -- (15.75pt, -729pt)
            ;
            \draw[color_29791,line width=1.5pt]
            (549.55pt, -15.59998pt) -- (549.55pt, -729pt)
            ;
            \draw[color_29791,line width=1.5pt]
            (15.75pt, -729pt) -- (549.55pt, -729pt)
            ;
            \draw[color_29791,line width=1.5pt]
            (15pt, -14.84998pt) -- (550.3pt, -14.84998pt)
            ;
        \end{tikzpicture}
    \end{minipage}
\end{figure}

\newpage
\pagestyle{plain}




\begin{figure}
\section*{Zadání}
U předložených setrvačníků zjistěte hodnotu výsledné práce \(W\), kterou je setrvačník schopen po získání kinetické energie dodat do zátěže.
Vyneste do grafu rozběhové a brzdící křivky setrvačníku \(I = F (t)\), dále okamžité hodnoty výkonu setrvačníků \(P = F (t)\). 
V grafu rozběhové křivky identifikujte ztráty způsobené konstrukcí zařízení. 
Porovnejte velikosti výkonů dodaných setrvačníky vzhledem k odporové zátěži.
Zjistěte velikost momentu hybnosti a kinetické energie akumulované setrvačníky. 
Setrvačníky seřaďte dle množství akumulované energie. 
Definujte, které parametry setrvačníku jsou k akumulaci energie nejdůležitější.
Na základě hmotnosti a přiložené výkresové dokumentace setrvačníku zjistěte, z jakých materiálů jsou vyrobeny.
Hmotnosti setrvačníků: \(m1 = 9,3 kg; m2 = 8,25 kg; m3 = 3,05 kg; m4 = 3,055 kg\)
Hodnoty odporů zátěže: \(R1 = 1 Ω; R2 = 2,2 Ω; R3 = 4 Ω;\)

\subsection{Teoretický úvod}



\end{figure}


\end{document}