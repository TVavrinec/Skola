\documentclass{article}
\usepackage{graphicx}
\usepackage{wrapfig}
\usepackage{filecontents}
\usepackage{siunitx}
\usepackage[table]{xcolor}
\usepackage{float}
\usepackage{hyperref}

\usepackage{color} % balíček pro obarvování textů
\usepackage{xcolor}  % zapne možnost používání barev, mj. pro \definecolor
\usepackage{pgfplots} % http://www.chiark.greenend.org.uk/doc/texlive-doc/latex/pgfplots/pgfplots.pdf
\pgfplotsset{compat=1.10}

\ifnum 0\ifxetex 1\fi\ifluatex 1\fi=0 % if pdftex
  \usepackage[T1]{fontenc}
  \usepackage[utf8]{inputenc}
\else % if luatex or xelatex
  \ifxetex
    \usepackage{mathspec}
  \else
    \usepackage{fontspec}
  \fi
  \defaultfontfeatures{Ligatures=TeX,Scale=MatchLowercase}
\fi
\usepackage[total={175mm,230mm}, top=23mm, left=20mm, includefoot]{geometry}
\hypersetup{
    colorlinks,
    linkcolor={blue!50!black},
    citecolor={green!50!black},
    urlcolor={blue!80!black}
}
% \definecolor{fialova}{RGB}{ 255, 000, 255}
\definecolor{color-si1}{RGB}{ 255, 000, 000}
\definecolor{color-si2}{RGB}{ 251, 130, 032}

\definecolor{color-ge1}{RGB}{ 000, 255, 000}
\definecolor{color-ge2}{RGB}{ 032, 251, 160}

\definecolor{color-inp1}{RGB}{ 000, 000, 255}
\definecolor{color-inp2}{RGB}{ 160, 032, 251}

\definecolor{color-geas1}{RGB}{ 225, 225, 000}
\definecolor{color-geas2}{RGB}{ 225, 225, 100}

\definecolor{sedak}{RGB}{ 100, 100, 100}


\newcommand \obr[1]
{ obr.~\ref{#1}}

\newcommand \tab[1]
{ tab.~ß\ref{#1}}


\begin{document}

\pagestyle{empty}

\definecolor{color_29791}{rgb}{0,0,0}
\begin{figure}[H]
    \hspace{-13mm}
    \begin{minipage}[t]{\textwidth}
        \vspace{-20mm}
        \begin{tikzpicture}[overlay]
            \path(0pt,0pt);
        \end{tikzpicture}
        \begin{picture}(-5,0)(2.5,0)
            \put(123.656,-82.75397){\fontsize{18}{1}\usefont{T1}{ptm}{m}{n}\selectfont\color{color_29791}VYSOKÉ UČENÍ TECHNICKÉ V BRNĚ}
            \put(76.296,-104.785){\fontsize{13}{1}\usefont{T1}{ptm}{m}{n}\selectfont\color{color_29791}FAKULTA  ELEKTROTECHNIKY A KOMUNIKAČNÍCH TECHNOLOGIÍ}
            \put(198.447,-128.5339){\fontsize{16}{1}\usefont{T1}{cmr}{b}{n}\selectfont\color{color_29791}Ústav elektrotechnologie}
            \put(156.848,-278.1589){\fontsize{14}{1}\usefont{T1}{ptm}{m}{n}\selectfont\color{color_29791}LABORATORNÍ CVIČENÍ Z PŘEDMĚTU}
            \put(108.123,-300.2579){\fontsize{14}{1}\usefont{T1}{cmr}{b}{n}\selectfont\color{color_29791}VYBRANÉ PARTIE Z OBNOVITELNÝCH ZDROJŮ A}
            \put(173.123,-320.2579){\fontsize{14}{1}\usefont{T1}{cmr}{b}{n}\selectfont\color{color_29791}UKLÁDÁNÍ ENERGIE (BPC-OZU)}
            \put(55.85,-421.25){\fontsize{14}{1}\usefont{T1}{cmr}{b}{n}\selectfont\color{color_29791}Číslo úlohy: 7}
            \put(55.85,-469.547){\fontsize{14}{1}\usefont{T1}{cmr}{b}{n}\selectfont\color{color_29791}Název úlohy: Využití termoelektrického jevu pro získávání energie}
            \put(23.9,-620.32){\fontsize{12}{1}\usefont{T1}{cmr}{b}{n}\selectfont\color{color_29791}Jméno a příjmení, ID:}
            \put(23.9,-637.119){\fontsize{12}{1}\usefont{T1}{cmr}{b}{n}\selectfont\color{color_29791}Tomáš Vavrinec, 240893}
            \put(186.95,-620.32){\fontsize{12}{1}\usefont{T1}{cmr}{b}{n}\selectfont\color{color_29791}Atmosférický tlak:}
            \put(186.95,-637.119){\fontsize{12}{1}\usefont{T1}{cmr}{b}{n}\selectfont\color{color_29791}1018 hPa}
            \put(293.25,-620.32){\fontsize{12}{1}\usefont{T1}{cmr}{b}{n}\selectfont\color{color_29791}Teplota okolí: }
            \put(293.25,-637.119){\fontsize{12}{1}\usefont{T1}{cmr}{b}{n}\selectfont\color{color_29791}21.7°C}
            \put(417.25,-620.32){\fontsize{12}{1}\usefont{T1}{cmr}{b}{n}\selectfont\color{color_29791}Relativní vlhkost:}
            \put(417.25,-637.119){\fontsize{12}{1}\usefont{T1}{cmr}{b}{n}\selectfont\color{color_29791}24.6\%}
            \put(23.9,-665.77){\fontsize{12}{1}\usefont{T1}{cmr}{b}{n}\selectfont\color{color_29791}Měřeno dne:}
            \put(23.9,-682.569){\fontsize{12}{1}\usefont{T1}{cmr}{b}{n}\selectfont\color{color_29791}25.2.2023}
            \put(186.95,-665.77){\fontsize{12}{1}\usefont{T1}{cmr}{b}{n}\selectfont\color{color_29791}Odevzdáno dne:}
            \put(293.25,-665.77){\fontsize{12}{1}\usefont{T1}{cmr}{b}{n}\selectfont\color{color_29791}Ročník, stud. skupina:}
            \put(293.25,-682.569){\fontsize{12}{1}\usefont{T1}{cmr}{b}{n}\selectfont\color{color_29791}2}
            \put(417.25,-665.77){\fontsize{12}{1}\usefont{T1}{cmr}{b}{n}\selectfont\color{color_29791}Kontrola:}
            \put(23.9,-703.42){\fontsize{12}{1}\usefont{T1}{cmr}{b}{n}\selectfont\color{color_29791}Spolupracovali:}
            \put(23.9,-720.219){\fontsize{12}{1}\usefont{T1}{cmr}{b}{n}\selectfont\color{color_29791}Kateřina Koudelková}
        \end{picture}
        \begin{tikzpicture}[overlay]
            \path(0pt,0pt);
            \draw[color_29791,line width=0.5pt]
            (20.4pt, -606.117pt) -- (20.4pt, -722.815pt)
            ;
            \draw[color_29791,line width=0.5pt]
            (183.45pt, -606.117pt) -- (183.45pt, -651.067pt)
            ;
            \draw[color_29791,line width=0.5pt]
            (183.45pt, -651.567pt) -- (183.45pt, -688.717pt)
            ;
            \draw[color_29791,line width=0.5pt]
            (289.75pt, -606.117pt) -- (289.75pt, -651.067pt)
            ;
            \draw[color_29791,line width=0.5pt]
            (289.75pt, -651.567pt) -- (289.75pt, -688.717pt)
            ;
            \draw[color_29791,line width=0.5pt]
            (413.75pt, -606.117pt) -- (413.75pt, -651.067pt)
            ;
            \draw[color_29791,line width=0.5pt]
            (413.75pt, -651.567pt) -- (413.75pt, -688.717pt)
            ;
            \draw[color_29791,line width=0.5pt]
            (544.9pt, -606.117pt) -- (544.9pt, -722.815pt)
            ;
            \draw[color_29791,line width=0.5pt]
            (20.15pt, -605.867pt) -- (545.15pt, -605.867pt)
            ;
            \draw[color_29791,line width=0.5pt]
            (20.65pt, -651.317pt) -- (544.65pt, -651.317pt)
            ;
            \draw[color_29791,line width=0.5pt]
            (20.65pt, -688.967pt) -- (544.65pt, -688.967pt)
            ;
            \draw[color_29791,line width=0.5pt]
            (20.15pt, -723.065pt) -- (545.15pt, -723.065pt)
            ;
            \draw[color_29791,line width=1.5pt]
            (15.75pt, -15.59998pt) -- (15.75pt, -729pt)
            ;
            \draw[color_29791,line width=1.5pt]
            (549.55pt, -15.59998pt) -- (549.55pt, -729pt)
            ;
            \draw[color_29791,line width=1.5pt]
            (15.75pt, -729pt) -- (549.55pt, -729pt)
            ;
            \draw[color_29791,line width=1.5pt]
            (15pt, -14.84998pt) -- (550.3pt, -14.84998pt)
            ;
        \end{tikzpicture}
    \end{minipage}
\end{figure}

\newpage
\pagestyle{plain}

\begin{minipage}[t]{\textwidth}
  \section*{Zadání}
  U předloženého systému vodních lázní s termočlánkem proměřte závislosti termoelektrického napětí na rozdílu teplot ve vodních lázních po stranách termočlánku.
  Výsledné závislosti změřených termoelektrických napětí na teplotě vyneste do grafu.
  Předložený systém vodních lázní s termočlánkem připojte ke zdroji a postupně zvyšujte napětí a pomocí přiložený teploměrů sledujte teploty v jednotlivých lázních.
  Výsledné závislosti změřených teplot na napětí přiloženém na termočlánek vyneste do grafu.
\end{minipage}


\section{Měření}

\subsection{úkol 1}
\begin{minipage}[t]{\textwidth}
  \centering
  \begin{tikzpicture}
    \begin{axis}[
      % axis y line*=left,
      width=\textwidth, 
      height=0.6\textwidth,
      title={Závislost generovaného výkonu na rozdílu teplot},
      xlabel={rozdíl teplot\(\Delta t\-[^\circ C]\)},
      ylabel={generovaný výkon \(P\-[mW]\)},
      xmin=10, xmax=110,
      ymin=0, ymax=2,
      legend pos=north west
      ]
      \addplot[
          % mark=x,
          color=blue,
        ]
        coordinates {
          (  20, 0.10)
          (45.5, 0.52)
          (84.2, 1.56)
          (100 , 1.76)
        };
      \addlegendentry{\(n\)}
    \end{axis}
  \end{tikzpicture}

  \begin{table}[H]
    \centering
    \begin{tabular}{|c|c|c|c|c|c|c|}
      \hline
      teplota chladné kádinky \(T\-[^\circ C]\) & \(27\)    & \(24,5\)    & \(0,8\)     & \(-24\)    \\ \hline
      teplota teplé kádinky \(T\-[^\circ C]\)	  & \(47\)	  & \(70\)	    & \(85\)	    & \(76\)	   \\ \hline
      rozdíl teplot\(\Delta T\-[^\circ C]\)	    & \(20\)	  & \(45,5\)    & \(84,2\)    & \(100\)	   \\ \hline
      generovaný proud \(I\-[mA]\)              & \(0,35\)  & \(0,75\)    & \(1,30\)    & \(1,40\)   \\ \hline
      generované napětí \(U\-[V]\)              & \(0,30\)  & \(0,70\)    & \(1,20\)    & \(1,26\)   \\ \hline
      Seebeckův koeficient \(\alpha [VK^{-1}]\) & \(0,015\) & \(0,01538\) & \(0,01425\) & \(0,0126\) \\ \hline
      generovaný výkon \(P\-[mW]\)              & \(0,10\)  & \(0,52\)    & \(1,56\)    & \(1,76\)   \\ \hline
    \end{tabular}
    \caption{\label{tabulka_mereni} První úloha}
  \end{table}
\end{minipage}

Příklady výpočtů: \\
\\
\large
\(
  \alpha = \frac{U}{\Delta T} = (\frac{0,3}{20})\-[VK^{-1}] = 0,015\-[VK^{-1}] \\ \\
  P = UI = (0,3 \cdot 0,35\cdot10^{-3})\-[W] = 0,1\cdot10^{-3}\-[W] = 0,1\-[mW]  \\ \\
\)
\normalsize \\

\vspace{-10mm}
\subsection{úkol 2}
\begin{minipage}[t]{\textwidth}
  \centering
  \begin{tikzpicture}
    \begin{axis}[
        width=\textwidth, 
        height=0.7\textwidth,
        title={Růst a pokles teploty vody při vytváření teplotního gradientu externím zdrojem},
        xlabel={\(n_{gen}\-[ot/min]\)},
        ylabel={\(U_{out_P-P}\-[V]\)},
        xmin=0, xmax=300,
        ymin=25, ymax=45,
        legend pos=north west
      ]
      \addplot[
        mark=x,
        color=blue
      ]
      coordinates {
        (  0, 39.2)
        ( 30, 38.2)
        ( 60, 37.9)
        ( 90, 37.4)
        (120, 37.1)
        (150, 36.7)
        (180, 36.4)
        (210, 36.0)
        (240, 35.7)
        (270, 35.4)
        (300, 35.1)
      };
      \addlegendentry{\(5V-T_1\)}
      \addplot[
      color=green,
      mark=x,
      ]
      coordinates {
        (  0, 39.8)
        ( 30, 39.9)
        ( 60, 40.0)
        ( 90, 40.0)
        (120, 40.0)
        (150, 40.0)
        (180, 40.0)
        (210, 40.0)
        (240, 40.0)
        (270, 40.1)
        (300, 40.1)
      };
      \addlegendentry{\(5V-T_2\)}
      \addplot[
        mark=x,
        color=red
      ]
      coordinates {
        (  0, 27.1)
        ( 30, 27.1)
        ( 60, 26.2)
        ( 90, 25.7)
        (120, 25.5)
        (150, 25.8)
        (180, 25.7)
        (210, 25.3)
        (240, 25.4)
        (270, 25.3)
        (300, 25.3)
      };
      \addlegendentry{\(10V-T_1\)}
      \addplot[
      color=orange,
      mark=x,
      ]
      coordinates {
        (  0, 27.3)
        ( 30, 27.4)
        ( 60, 27.7)
        ( 90, 28.0)
        (120, 28.0)
        (150, 28.1)
        (180, 28.3)
        (210, 28.5)
        (240, 28.8)
        (270, 29.1)
        (300, 29.5)
      };
      \addlegendentry{\(10V-T_2\)}
    \end{axis}
  \end{tikzpicture}

  \begin{table}[H]
    \centering
    \begin{tabular}{|c|c|c|c|c|c|c|c|c|c|c|c|}
      \hline
      \(t\-[s]\)	                & \(0\)     & \(30\)    & \(60\)    & \(90\)    & \(120\)   & \(150\)   & \(180\)   & \(210\)   & \(240\)   & \(270\)   & \(300\)   \\ \hline
      \(T_{1}\-[^\circ C]\)       & \(39,2\)  & \(38,2\)  & \(37,9\)  & \(37,4\)  & \(37,1\)  & \(36,7\)  & \(36,4\)  & \(36,0\)  & \(35,7\)  & \(35,4\)  & \(35,1\)  \\ \hline
      \(T_{2}\-[^\circ C]\)       & \(39,8\)  & \(39,9\)  & \(40,0\)  & \(40,0\)  & \(40,0\)  & \(40,0\)  & \(40,0\)  & \(40,0\)  & \(40,0\)  & \(40,1\)  & \(40,1\)  \\ \hline
      \(P\-[W]\)                  &           &           & \(19,5\)  & \(24,2\)  & \(20,2\)  & \(18,4\)  & \(16,7\)  & \(15,9\)  & \(15,0\)  & \(14,6\)  & \(13,9\)  \\ \hline
    \end{tabular}
    \caption{\label{tabulka_mereni-5v} 5V napájení - \(I=1.3\-[A]\)}
  \end{table}
  \begin{table}[H]
    \centering
    \begin{tabular}{|c|c|c|c|c|c|c|c|c|c|c|c|}
      \hline
      \(t\-[s]\)	                & \(0\)     & \(30\)    & \(60\)    & \(90\)    & \(120\)   & \(150\)   & \(180\)   & \(210\)   & \(240\)   & \(270\)   & \(300\)   \\ \hline
      \(T_{1}\-[^\circ C]\)       & \(27,1\)  & \(27,1\)  & \(26,2\)  & \(25,7\)  & \(25,5\)  & \(25,8\)  & \(25,7\)  & \(25,3\)  & \(25,4\)  & \(25,3\)  & \(25,3\)  \\ \hline
      \(T_{2}\-[^\circ C]\)       & \(27,3\)  & \(27,4\)  & \(27,7\)  & \(28,0\)  & \(28,0\)  & \(28,1\)  & \(28,3\)  & \(28,5\)  & \(28,8\)  & \(29,1\)  & \(29,5\)  \\ \hline
      \(P\-[W]\)                  &           &           & \(20,9\)  & \(21,4\)  & \(17,4\)  & \(12,8\)  & \(12,1\)  & \(12,7\)  & \(11,8\)  & \(11,8\)  & \(11,7\)  \\ \hline
    \end{tabular}
    \caption{\label{tabulka_mereni-10v} 10V napájení - \(I=2,53\-[A]\)}
  \end{table}
\end{minipage}
Příklady výpočtů: \\
\\
\large
\(
  P = \frac{m \cdot C_v (t_2 - t_1)}{\tau} = \frac{200 \cdot 10^{-3} \cdot 4180 (40,1 - 35,1)}{300} = 13,9\-[W]
\)
\normalsize \\

\newpage
Zjistěte účinnost přeměny tepelné energie na energii elektrickou, jestliže víme, že MARS Curiosity používá ke generaci elektrické energie zdroj tepla o výkonu \(P_{tep} = 2000\-[W]\) a baterie s napětím \(U = 7,2\-[V]\) 
  a kapacitou \(C = 43\-[Ah]\) nabije při využití celkové generované energie termočlánkem za \(t = 2,5\-[h]\).\\
\\
\large
\(
  W_{bat} = U\cdot C = (7,2 \cdot 43 \cdot 3600)\-[J] = 1114560\-[J] \\ \\
  W_{tep} = P_{tep} \cdot t = (2000 \cdot 2,5 \cdot 3600)\-[J] = 18000000\-[J] \\ \\
  \eta = \frac{W_{bat}}{W_{tep}} = \frac{1114560}{18000000} = 0,062 \Rightarrow \eta = 6,2\-[\%] \\ \\
\)
\normalsize \\


\subsection{Závěr}
Peltiéruv jev je jev, při kterém působením teplotního gradientu vzniká napětí na kontaktu dvou vodičů. Seebeckův jev je naopak jev, při kterém na přechodu mezi dvěma kovy vzniká teplotní gradient působením vnějšího elektrického zdroje.

Účinnost přeměny tepelné energie na elektrickou pomocí Peltiérova jevu na Curiosity je \(6.2\-[\%]\).

\end{document}