\documentclass{article}
\usepackage{graphicx}
\usepackage{wrapfig}
\usepackage{filecontents}
\usepackage{siunitx}
\usepackage[table]{xcolor}
\usepackage{float}
\usepackage{hyperref}

\usepackage{color} % balíček pro obarvování textů
\usepackage{xcolor}  % zapne možnost používání barev, mj. pro \definecolor
\usepackage{pgfplots} % http://www.chiark.greenend.org.uk/doc/texlive-doc/latex/pgfplots/pgfplots.pdf
\pgfplotsset{compat=1.10}

\ifnum 0\ifxetex 1\fi\ifluatex 1\fi=0 % if pdftex
  \usepackage[T1]{fontenc}
  \usepackage[utf8]{inputenc}
\else % if luatex or xelatex
  \ifxetex
    \usepackage{mathspec}
  \else
    \usepackage{fontspec}
  \fi
  \defaultfontfeatures{Ligatures=TeX,Scale=MatchLowercase}
\fi
\usepackage[total={175mm,230mm}, top=23mm, left=20mm, includefoot]{geometry}
\hypersetup{
    colorlinks,
    linkcolor={blue!50!black},
    citecolor={green!50!black},
    urlcolor={blue!80!black}
}
% \definecolor{fialova}{RGB}{ 255, 000, 255}
\definecolor{color-si1}{RGB}{ 255, 000, 000}
\definecolor{color-si2}{RGB}{ 251, 130, 032}

\definecolor{color-ge1}{RGB}{ 000, 255, 000}
\definecolor{color-ge2}{RGB}{ 032, 251, 160}

\definecolor{color-inp1}{RGB}{ 000, 000, 255}
\definecolor{color-inp2}{RGB}{ 160, 032, 251}

\definecolor{color-geas1}{RGB}{ 225, 225, 000}
\definecolor{color-geas2}{RGB}{ 225, 225, 100}

\definecolor{sedak}{RGB}{ 100, 100, 100}


\newcommand \obr[1]
{ obr.~\ref{#1}}

\newcommand \tab[1]
{ tab.~ß\ref{#1}}


\begin{document}

\pagestyle{empty}

\definecolor{color_29791}{rgb}{0,0,0}
\begin{figure}[H]
    \hspace{-13mm}
    \begin{minipage}[t]{\textwidth}
        \vspace{-20mm}
        \begin{tikzpicture}[overlay]
            \path(0pt,0pt);
        \end{tikzpicture}
        \begin{picture}(-5,0)(2.5,0)
            \put(123.656,-82.75397){\fontsize{18}{1}\usefont{T1}{ptm}{m}{n}\selectfont\color{color_29791}VYSOKÉ UČENÍ TECHNICKÉ V BRNĚ}
            \put(76.296,-104.785){\fontsize{13}{1}\usefont{T1}{ptm}{m}{n}\selectfont\color{color_29791}FAKULTA  ELEKTROTECHNIKY A KOMUNIKAČNÍCH TECHNOLOGIÍ}
            \put(198.447,-128.5339){\fontsize{16}{1}\usefont{T1}{cmr}{b}{n}\selectfont\color{color_29791}Ústav elektrotechnologie}
            \put(156.848,-278.1589){\fontsize{14}{1}\usefont{T1}{ptm}{m}{n}\selectfont\color{color_29791}LABORATORNÍ CVIČENÍ Z PŘEDMĚTU}
            \put(108.123,-300.2579){\fontsize{14}{1}\usefont{T1}{cmr}{b}{n}\selectfont\color{color_29791}VYBRANÉ PARTIE Z OBNOVITELNÝCH ZDROJŮ A}
            \put(173.123,-320.2579){\fontsize{14}{1}\usefont{T1}{cmr}{b}{n}\selectfont\color{color_29791}UKLÁDÁNÍ ENERGIE (BPC-OZU)}
            \put(55.85,-421.25){\fontsize{14}{1}\usefont{T1}{cmr}{b}{n}\selectfont\color{color_29791}Číslo úlohy: 7}
            \put(55.85,-469.547){\fontsize{14}{1}\usefont{T1}{cmr}{b}{n}\selectfont\color{color_29791}Název úlohy: Využití termoelektrického jevu pro získávání energie}
            \put(23.9,-620.32){\fontsize{12}{1}\usefont{T1}{cmr}{b}{n}\selectfont\color{color_29791}Jméno a příjmení, ID:}
            \put(23.9,-637.119){\fontsize{12}{1}\usefont{T1}{cmr}{b}{n}\selectfont\color{color_29791}Tomáš Vavrinec, 240893}
            \put(186.95,-620.32){\fontsize{12}{1}\usefont{T1}{cmr}{b}{n}\selectfont\color{color_29791}Atmosférický tlak:}
            \put(186.95,-637.119){\fontsize{12}{1}\usefont{T1}{cmr}{b}{n}\selectfont\color{color_29791}1018 hPa}
            \put(293.25,-620.32){\fontsize{12}{1}\usefont{T1}{cmr}{b}{n}\selectfont\color{color_29791}Teplota okolí: }
            \put(293.25,-637.119){\fontsize{12}{1}\usefont{T1}{cmr}{b}{n}\selectfont\color{color_29791}21.7°C}
            \put(417.25,-620.32){\fontsize{12}{1}\usefont{T1}{cmr}{b}{n}\selectfont\color{color_29791}Relativní vlhkost:}
            \put(417.25,-637.119){\fontsize{12}{1}\usefont{T1}{cmr}{b}{n}\selectfont\color{color_29791}24.6\%}
            \put(23.9,-665.77){\fontsize{12}{1}\usefont{T1}{cmr}{b}{n}\selectfont\color{color_29791}Měřeno dne:}
            \put(23.9,-682.569){\fontsize{12}{1}\usefont{T1}{cmr}{b}{n}\selectfont\color{color_29791}25.2.2023}
            \put(186.95,-665.77){\fontsize{12}{1}\usefont{T1}{cmr}{b}{n}\selectfont\color{color_29791}Odevzdáno dne:}
            \put(293.25,-665.77){\fontsize{12}{1}\usefont{T1}{cmr}{b}{n}\selectfont\color{color_29791}Ročník, stud. skupina:}
            \put(293.25,-682.569){\fontsize{12}{1}\usefont{T1}{cmr}{b}{n}\selectfont\color{color_29791}2}
            \put(417.25,-665.77){\fontsize{12}{1}\usefont{T1}{cmr}{b}{n}\selectfont\color{color_29791}Kontrola:}
            \put(23.9,-703.42){\fontsize{12}{1}\usefont{T1}{cmr}{b}{n}\selectfont\color{color_29791}Spolupracovali:}
            \put(23.9,-720.219){\fontsize{12}{1}\usefont{T1}{cmr}{b}{n}\selectfont\color{color_29791}Kateřina Koudelková}
        \end{picture}
        \begin{tikzpicture}[overlay]
            \path(0pt,0pt);
            \draw[color_29791,line width=0.5pt]
            (20.4pt, -606.117pt) -- (20.4pt, -722.815pt)
            ;
            \draw[color_29791,line width=0.5pt]
            (183.45pt, -606.117pt) -- (183.45pt, -651.067pt)
            ;
            \draw[color_29791,line width=0.5pt]
            (183.45pt, -651.567pt) -- (183.45pt, -688.717pt)
            ;
            \draw[color_29791,line width=0.5pt]
            (289.75pt, -606.117pt) -- (289.75pt, -651.067pt)
            ;
            \draw[color_29791,line width=0.5pt]
            (289.75pt, -651.567pt) -- (289.75pt, -688.717pt)
            ;
            \draw[color_29791,line width=0.5pt]
            (413.75pt, -606.117pt) -- (413.75pt, -651.067pt)
            ;
            \draw[color_29791,line width=0.5pt]
            (413.75pt, -651.567pt) -- (413.75pt, -688.717pt)
            ;
            \draw[color_29791,line width=0.5pt]
            (544.9pt, -606.117pt) -- (544.9pt, -722.815pt)
            ;
            \draw[color_29791,line width=0.5pt]
            (20.15pt, -605.867pt) -- (545.15pt, -605.867pt)
            ;
            \draw[color_29791,line width=0.5pt]
            (20.65pt, -651.317pt) -- (544.65pt, -651.317pt)
            ;
            \draw[color_29791,line width=0.5pt]
            (20.65pt, -688.967pt) -- (544.65pt, -688.967pt)
            ;
            \draw[color_29791,line width=0.5pt]
            (20.15pt, -723.065pt) -- (545.15pt, -723.065pt)
            ;
            \draw[color_29791,line width=1.5pt]
            (15.75pt, -15.59998pt) -- (15.75pt, -729pt)
            ;
            \draw[color_29791,line width=1.5pt]
            (549.55pt, -15.59998pt) -- (549.55pt, -729pt)
            ;
            \draw[color_29791,line width=1.5pt]
            (15.75pt, -729pt) -- (549.55pt, -729pt)
            ;
            \draw[color_29791,line width=1.5pt]
            (15pt, -14.84998pt) -- (550.3pt, -14.84998pt)
            ;
        \end{tikzpicture}
    \end{minipage}
\end{figure}

\newpage
\pagestyle{plain}

\begin{minipage}[t]{\textwidth}
  \section*{Zadání}
  Určete typ použitého Stirlingova motoru. 
  Pomocí topného elementu zahřejte výměník tepla a sledujte napětí a proud generovaný Stirlingovým motorem, příkon a teplotu na výměníku při různých hodnotách proudu. 
  Následně budou vypočteny výkony a účinnost pro jednotlivé příkony a bude sestaven graf výkonu motoru v závislosti na dosažené teplotě a příkonu.
  V druhé části úlohy připojíte ke hřídeli elektromotor a přiložením napětí ji roztočíte a budete sledovat, jak Stirlingův motor funguje jako tepelné čerpadlo při různých způsobech otáčení.
\end{minipage}


\section{Měření}

\subsection{úkol 1}
\begin{minipage}[t]{\textwidth}
  \centering
  \begin{tikzpicture}
    \begin{axis}[
      % axis y line*=left,
      width=\textwidth, 
      height=0.6\textwidth,
      title={Závislost výstupního výkonu na vstupním},
      xlabel={vstupní výkon \(P_{in}\-[W]\)},
      ylabel={Účinnost \(\eta\-[\text{\textperthousand}]\)},
      xmin=80, xmax=180,
      ymin=0, ymax=0.02,
      legend pos=north west
      ]
      \addplot[
          % mark=x,
          color=blue,
        ]
        coordinates {
          (  82.6, 0.002)
          (  91.9, 0.005)
          ( 111.1, 0.012)
          ( 129.6, 0.017)
          ( 150.8, 0.016)
          ( 175.0, 0.002)
        };
      % \addlegendentry{\(n\)}
    \end{axis}
  \end{tikzpicture}

  \begin{table}[H]
    \centering
    \begin{tabular}{|c|c|c|c|c|c|c|}
      \hline
      Proud do topného tělesa \(I_{in}\-[A]\)       & \(9,5\)           & \(10,1\)         & \(11\)             & \(12\)            & \(13\)            & \(14\)            \\ \hline
      Napětí na topném tělese \(U_{in}\-[V]\)       & \(8,7\)           & \(9,1\)          & \(10,1\)           & \(10,8\)          & \(11,6\)          & \(12,5\)          \\ \hline
      Generovaný proud \(I_{out}\-[mA]\)            & \(0,45\)          & \(0,475\)        & \(1,1\)            & \(1,45\)          & \(1,6\)           & \(0,8\)           \\ \hline
      Generované napětí \(U\-[V]\)                  & \(0,4\)           & \(0,9\)          & \(1,2\)            & \(1,5\)           & \(1,5\)           & \(0,5\)           \\ \hline \hline
      Příkon \(P_{in}\-[W]\)                        & \(82,6\)          & \(91,9\)         & \(111,1\)          & \(129,6\)         & \(150,8\)         & \(175\)           \\ \hline
      Výkon \(P_{out}\-[mW]\)                       & \(0,18\)          & \(0,428\)        & \(1,32\)           & \(2,175\)         & \(2,4\)           & \(0,4\)           \\ \hline
      Účinnost \(\eta\-[\text{\textperthousand}]\)  & \(2\cdot10{-3}\)  & \(5\cdot10{-3}\) & \(12\cdot10{-3}\)  & \(17\cdot10{-3}\) & \(16\cdot10{-3}\) & \(2\cdot10{-3}\)  \\ \hline
    \end{tabular}
    \caption{\label{tabulka_mereni} První úloha}
  \end{table}
\end{minipage}

\newpage

Příklady výpočtů: \\
\\
\large
\(
  P_{in} = I_{in} \cdot U_{in} = (9,5 \cdot 8,7)\-[W] = 82,6\-[W] \\ \\
  P_{out} = I_{out} \cdot U_{out} = (0,45\cdot 10^{-3} \cdot 0,4)\-[W] = 0,18^{-3}\-[W] = 0,18\-[mW] \\ \\
  \eta = \frac{P_{out}}{P_{in}} = \frac{0,18\cdot{-3}}{82,6} = 2,179\cdot10^{-6} = 2,179\cdot10{-3}\-[\text{\textperthousand}] \\ \\
\)
\large \\

\vspace{-10mm}
\subsection{úkol 2}
\begin{minipage}[t]{\textwidth}
  \begin{table}[H]
    \centering
    \begin{tabular}{|c|c|c|c|c|c|c|c|c|c|c|c|}
      \hline
                                                        & \(80^\circ C\)  & \(40^\circ C\)  \\ \hline
      Doba chladnutí bez napájení \([s]\)	              & \(510\)         & \(1140\)        \\ \hline
      Doba chladnutí s napájením \(10\-[V]\) - \([s]\)  & \(476\)         & \(1032\)        \\ \hline
    \end{tabular}
    \caption{\label{tabulka_mereni-5v} Ochlazování motoru po zahřátí na \(300^\circ C\)}
  \end{table}
\end{minipage}
Poslední měření jsme nemohli provést, protože při posledním zahřívání se vysunulo těsnění a motor tak nebyl schopen dalšího chodu.

\subsection{Závěr}
Měřený Stirlingův motor byla modifikace \(\alpha\)

Námi naměřená účinnost Stirlingova motoru se blíží nule \\ (maximální měřená účinnost \(\eta = 16\cdot10{-3}\-[\text{\textperthousand}]\)).
Tato účinnost je ale účinnost celého systému, kde Stirlingův motor není sám, pravděpodobně však byla jeho účinnost i tak extrémně malá.


\end{document}