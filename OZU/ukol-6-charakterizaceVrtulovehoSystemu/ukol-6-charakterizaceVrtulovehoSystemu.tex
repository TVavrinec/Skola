\documentclass{article}
\usepackage{graphicx}
\usepackage{wrapfig}
\usepackage{filecontents}
\usepackage{siunitx}
\usepackage[table]{xcolor}
\usepackage{float}
\usepackage{hyperref}

\usepackage{color} % balíček pro obarvování textů
\usepackage{xcolor}  % zapne možnost používání barev, mj. pro \definecolor
\usepackage{pgfplots} % http://www.chiark.greenend.org.uk/doc/texlive-doc/latex/pgfplots/pgfplots.pdf
\pgfplotsset{compat=1.10}

\ifnum 0\ifxetex 1\fi\ifluatex 1\fi=0 % if pdftex
  \usepackage[T1]{fontenc}
  \usepackage[utf8]{inputenc}
\else % if luatex or xelatex
  \ifxetex
    \usepackage{mathspec}
  \else
    \usepackage{fontspec}
  \fi
  \defaultfontfeatures{Ligatures=TeX,Scale=MatchLowercase}
\fi
\usepackage[total={175mm,230mm}, top=23mm, left=20mm, includefoot]{geometry}
\hypersetup{
    colorlinks,
    linkcolor={blue!50!black},
    citecolor={green!50!black},
    urlcolor={blue!80!black}
}
% \definecolor{fialova}{RGB}{ 255, 000, 255}
\definecolor{color-si1}{RGB}{ 255, 000, 000}
\definecolor{color-si2}{RGB}{ 251, 130, 032}

\definecolor{color-ge1}{RGB}{ 000, 255, 000}
\definecolor{color-ge2}{RGB}{ 032, 251, 160}

\definecolor{color-inp1}{RGB}{ 000, 000, 255}
\definecolor{color-inp2}{RGB}{ 160, 032, 251}

\definecolor{color-geas1}{RGB}{ 225, 225, 000}
\definecolor{color-geas2}{RGB}{ 225, 225, 100}

\definecolor{sedak}{RGB}{ 100, 100, 100}


\newcommand \obr[1]
{ obr.~\ref{#1}}

\newcommand \tab[1]
{ tab.~ß\ref{#1}}


\begin{document}

\pagestyle{empty}

\definecolor{color_29791}{rgb}{0,0,0}
\begin{figure}[H]
    \hspace{-13mm}
    \begin{minipage}[t]{\textwidth}
        \vspace{-20mm}
        \begin{tikzpicture}[overlay]
            \path(0pt,0pt);
        \end{tikzpicture}
        \begin{picture}(-5,0)(2.5,0)
            \put(123.656,-82.75397){\fontsize{18}{1}\usefont{T1}{ptm}{m}{n}\selectfont\color{color_29791}VYSOKÉ UČENÍ TECHNICKÉ V BRNĚ}
            \put(76.296,-104.785){\fontsize{13}{1}\usefont{T1}{ptm}{m}{n}\selectfont\color{color_29791}FAKULTA  ELEKTROTECHNIKY A KOMUNIKAČNÍCH TECHNOLOGIÍ}
            \put(198.447,-128.5339){\fontsize{16}{1}\usefont{T1}{cmr}{b}{n}\selectfont\color{color_29791}Ústav elektrotechnologie}
            \put(156.848,-278.1589){\fontsize{14}{1}\usefont{T1}{ptm}{m}{n}\selectfont\color{color_29791}LABORATORNÍ CVIČENÍ Z PŘEDMĚTU}
            \put(108.123,-300.2579){\fontsize{14}{1}\usefont{T1}{cmr}{b}{n}\selectfont\color{color_29791}VYBRANÉ PARTIE Z OBNOVITELNÝCH ZDROJŮ A}
            \put(173.123,-320.2579){\fontsize{14}{1}\usefont{T1}{cmr}{b}{n}\selectfont\color{color_29791}UKLÁDÁNÍ ENERGIE (BPC-OZU)}
            \put(55.85,-421.25){\fontsize{14}{1}\usefont{T1}{cmr}{b}{n}\selectfont\color{color_29791}Číslo úlohy: 7}
            \put(55.85,-469.547){\fontsize{14}{1}\usefont{T1}{cmr}{b}{n}\selectfont\color{color_29791}Název úlohy: Využití termoelektrického jevu pro získávání energie}
            \put(23.9,-620.32){\fontsize{12}{1}\usefont{T1}{cmr}{b}{n}\selectfont\color{color_29791}Jméno a příjmení, ID:}
            \put(23.9,-637.119){\fontsize{12}{1}\usefont{T1}{cmr}{b}{n}\selectfont\color{color_29791}Tomáš Vavrinec, 240893}
            \put(186.95,-620.32){\fontsize{12}{1}\usefont{T1}{cmr}{b}{n}\selectfont\color{color_29791}Atmosférický tlak:}
            \put(186.95,-637.119){\fontsize{12}{1}\usefont{T1}{cmr}{b}{n}\selectfont\color{color_29791}1018 hPa}
            \put(293.25,-620.32){\fontsize{12}{1}\usefont{T1}{cmr}{b}{n}\selectfont\color{color_29791}Teplota okolí: }
            \put(293.25,-637.119){\fontsize{12}{1}\usefont{T1}{cmr}{b}{n}\selectfont\color{color_29791}21.7°C}
            \put(417.25,-620.32){\fontsize{12}{1}\usefont{T1}{cmr}{b}{n}\selectfont\color{color_29791}Relativní vlhkost:}
            \put(417.25,-637.119){\fontsize{12}{1}\usefont{T1}{cmr}{b}{n}\selectfont\color{color_29791}24.6\%}
            \put(23.9,-665.77){\fontsize{12}{1}\usefont{T1}{cmr}{b}{n}\selectfont\color{color_29791}Měřeno dne:}
            \put(23.9,-682.569){\fontsize{12}{1}\usefont{T1}{cmr}{b}{n}\selectfont\color{color_29791}25.2.2023}
            \put(186.95,-665.77){\fontsize{12}{1}\usefont{T1}{cmr}{b}{n}\selectfont\color{color_29791}Odevzdáno dne:}
            \put(293.25,-665.77){\fontsize{12}{1}\usefont{T1}{cmr}{b}{n}\selectfont\color{color_29791}Ročník, stud. skupina:}
            \put(293.25,-682.569){\fontsize{12}{1}\usefont{T1}{cmr}{b}{n}\selectfont\color{color_29791}2}
            \put(417.25,-665.77){\fontsize{12}{1}\usefont{T1}{cmr}{b}{n}\selectfont\color{color_29791}Kontrola:}
            \put(23.9,-703.42){\fontsize{12}{1}\usefont{T1}{cmr}{b}{n}\selectfont\color{color_29791}Spolupracovali:}
            \put(23.9,-720.219){\fontsize{12}{1}\usefont{T1}{cmr}{b}{n}\selectfont\color{color_29791}Kateřina Koudelková}
        \end{picture}
        \begin{tikzpicture}[overlay]
            \path(0pt,0pt);
            \draw[color_29791,line width=0.5pt]
            (20.4pt, -606.117pt) -- (20.4pt, -722.815pt)
            ;
            \draw[color_29791,line width=0.5pt]
            (183.45pt, -606.117pt) -- (183.45pt, -651.067pt)
            ;
            \draw[color_29791,line width=0.5pt]
            (183.45pt, -651.567pt) -- (183.45pt, -688.717pt)
            ;
            \draw[color_29791,line width=0.5pt]
            (289.75pt, -606.117pt) -- (289.75pt, -651.067pt)
            ;
            \draw[color_29791,line width=0.5pt]
            (289.75pt, -651.567pt) -- (289.75pt, -688.717pt)
            ;
            \draw[color_29791,line width=0.5pt]
            (413.75pt, -606.117pt) -- (413.75pt, -651.067pt)
            ;
            \draw[color_29791,line width=0.5pt]
            (413.75pt, -651.567pt) -- (413.75pt, -688.717pt)
            ;
            \draw[color_29791,line width=0.5pt]
            (544.9pt, -606.117pt) -- (544.9pt, -722.815pt)
            ;
            \draw[color_29791,line width=0.5pt]
            (20.15pt, -605.867pt) -- (545.15pt, -605.867pt)
            ;
            \draw[color_29791,line width=0.5pt]
            (20.65pt, -651.317pt) -- (544.65pt, -651.317pt)
            ;
            \draw[color_29791,line width=0.5pt]
            (20.65pt, -688.967pt) -- (544.65pt, -688.967pt)
            ;
            \draw[color_29791,line width=0.5pt]
            (20.15pt, -723.065pt) -- (545.15pt, -723.065pt)
            ;
            \draw[color_29791,line width=1.5pt]
            (15.75pt, -15.59998pt) -- (15.75pt, -729pt)
            ;
            \draw[color_29791,line width=1.5pt]
            (549.55pt, -15.59998pt) -- (549.55pt, -729pt)
            ;
            \draw[color_29791,line width=1.5pt]
            (15.75pt, -729pt) -- (549.55pt, -729pt)
            ;
            \draw[color_29791,line width=1.5pt]
            (15pt, -14.84998pt) -- (550.3pt, -14.84998pt)
            ;
        \end{tikzpicture}
    \end{minipage}
\end{figure}

\newpage
\pagestyle{plain}

\begin{minipage}[t]{\textwidth}
  \section*{Zadání}
  \begin{enumerate}
    \item Sestavte graf závislosti \(v_{air} = F(U)\) a \(n_t = F(U)\).
    \item Vypočtěte účinnost \(\eta\) zdroje větrné energie z dodaného příkonu do motorku a výstupní hustoty výkonu větru pro napájecí napětí \(U=6\-[V]\) a \(U=12\-[V]\) v ustáleném stavu. Účinnosti porovnejte.
    \item Sestavte graf závislosti výstupního napětí generátoru na otáčkách genterátoru \(U_{out}=F(n_g)\) v semilogaritmických souřadnicích.
    \item Vypočtěte počet pólových dvojic použitého stejnosměrného motoru pracující v generátorickém režimu.
    \item Stanovte startovací rychlost větru \(v_{air}\) zkoumaného systému.
  \end{enumerate}
\end{minipage}


\section{Měření}

\subsection{úkol 1}
\begin{minipage}[t]{\textwidth}
  \centering
  \begin{tikzpicture}
    \begin{axis}[
      axis y line*=left,
      width=\textwidth, 
      height=0.6\textwidth,
      title={Závislost rychlosti vzduch na otáčkách vrtule},
      xlabel={\(U\-[V]\)},
      ylabel={\(n\-[ot/min]\)},
      xmin=0, xmax=12,
      ymin=0, ymax=3200,
      legend pos=north west
      ]
      \addplot[
          % mark=x,
          color=blue,
        ]
        coordinates {
          (  0,    0)
          (  1,  260)
          (  2,  686)
          (  3,  996)
          (  4, 1332)
          (  5, 1638)
          (  6, 1902)
          (  7, 2155)
          (  8, 2390)
          (  9, 2583)
          ( 10, 2768)
          ( 11, 2947)
          ( 12, 3101)
        };
      \addlegendentry{\(n\)}
      \end{axis}
      \begin{axis}[
            % colorbar,
            width=\textwidth, 
            height=0.6\textwidth,
            axis x line=none,
            axis y line*=right,
            ylabel={\(V_{air}\-[ms^{-1}]\)},
            xmin=0, xmax=12,
            ymin=0, ymax=6,
            legend style={at={(0.13,0.9)}}
        ]
        \addplot[
          color=red,
          % mark=*,
          ]
          coordinates {
            (  0, 0.0)
            (  1, 0.0)
            (  2, 0.3)
            (  3, 0.9)
            (  4, 1.6)
            (  5, 2.2)
            (  6, 2.8)
            (  7, 3.2)
            (  8, 3.7)
            (  9, 4.1)
            ( 10, 4.7)
            ( 11, 4.9)
            ( 12, 5.2)
          };
        \addlegendentry{\(V_{air}\)}
    \end{axis}
  \end{tikzpicture}

  \begin{table}[H]
    \centering
    \begin{tabular}{|c|c|c|c|c|c|c|c|c|c|c|c|c|c|}
      \hline
      \(U\-[V]\)	            & \(0\)	& \(1\)   & \(2\)                 & \(3\)       & \(4\)	      & \(5\)     & \(6\)     & \(7\)     & \(8\)	    & \(9\)     & \(10\)    & \(11\)    & \(12\)    \\ \hline
      \(V_{air}\-[ms^{-1}]\)  & \(0\)	& \(0\)   & \(0.3\)               & \(0.9\)     & \(1.6\)	    & \(2.2\)   & \(2.8\)   & \(3.2\)   & \(3.7\)	  & \(4.1\)   & \(4.7\)   & \(4.9\)   & \(5.2\)   \\ \hline
      \(n\-[ot/min]\)	        & \(0\)	& \(260\) & \(686\)               & \(996\)     & \(1332\)    & \(1638\)  & \(1902\)  & \(2155\)  & \(2390\)  & \(2583\)  & \(2768\)  & \(2947\)  & \(3101\)  \\ \hline
      \(I\-[mA]\)             &       &         &                       &             &             &           & \(340\)	  &           &           &           &           &           & \(750\)   \\ \hline \hline
      \(P_V\-[W]\)            & \(0\) & \(0\)   & \(5.48\cdot10^{-4}\)  & \(0.0158\)  & \(0.0832\)  & \(0.216\) & \(0.446\) & \(0.666\) & \(1.029\) & \(1.401\) & \(2.110\) & \(2.391\) & \(2.857\) \\ \hline
      \(P_{IN}\-[W]\)         &       &         &                       &             &             &           & \(2.040\) &           &           &           &           &           & \(9.0\)   \\ \hline
      \(\eta\-[\%]\)          &       &         &                       &             &             &           & \(21.9\)  &           &           &           &           &           & \(31.7\)  \\ \hline
    \end{tabular}
    \caption{\label{tabulka_mereni} První úloha}
  \end{table}
\end{minipage}

Příklady výpočtů: \\
\\
\large
\(
  P_V = \frac{1}{2} \rho v^3 r^2\pi = (\frac{1}{2} 1.29371 5.2^3)\-[Wm^{-2}] \cdot (0.1^2\pi)\-[m^2]= 2.857\-[W] \\ \\
  P_{IN} = UI = (12 \cdot 750\cdot10^{-3})\-[W] = 9\-[W]  \\ \\
  \eta = \frac{P_V}{P_{IN}} \cdot 100\% = \frac{9}{2.857} \cdot 100\% = 31.7\-[\%] 
\)
\normalsize \\

\subsection{úkol 2}
\begin{minipage}[t]{\textwidth}
  \centering
  \begin{tikzpicture}
    \begin{axis}[
        axis y line*=left,
        width=\textwidth, 
        height=0.5\textwidth,
        title={Závislost rychlosti vzduch na otáčkách vrtule},
        xlabel={\(n_{gen}\-[ot/min]\)},
        ylabel={\(U_{out_P-P}\-[V]\)},
        xmin=0, xmax=2000,
        ymin=0, ymax=4,
        legend pos=north west
      ]
      \addplot[
        mark=x,
        color=blue
      ]
      coordinates {
        (   50, 1.060)
        (  167, 1.175)
        (  303, 1.560)
        (  518, 1.713)
        ( 1691, 2.625)
        ( 1919, 3.350)
      };
      \addlegendentry{\(U_{out_P-P}\)}
    \end{axis}
    \begin{axis}[
        % colorbar,
        width=\textwidth, 
        height=0.5\textwidth,
        axis x line=none,
        axis y line*=right,
        ylabel={\(V_{air}\-[ms^{-1}]\)},
        xmin=0, xmax=2000,
        ymin=0, ymax=6,
        legend style={at={(0.13,0.9)}}
      ]
      \addplot[
        color=red,
        mark=x,
        ]
        coordinates {
          (   50, 3.2)
          (  167, 3.7)
          (  303, 4.1)
          (  518, 4.7)
          ( 1691, 4.9)
          ( 1919, 5.2)
        };
      \addlegendentry{\(V_{air}\)}
    \end{axis}
  \end{tikzpicture}

  \begin{table}[H]
    \centering
    \begin{tabular}{|c|c|c|c|c|c|c|c|}
      \hline
      \(U\-[V]\)	                & \(7\)     & \(8\)	     & \(9\)    & \(10\)    & \(11\)     & \(12\)     \\ \hline
      \(U_{out_P-P}\-[V]\)        & \(1.06\)  & \(1.175\)  & \(1.56\) & \(1.713\) & \(2.625\)  & \(3.35\)   \\ \hline
      \(n_{primar}\-[ot/min]\)    & \(2155\)  & \(2390\)   & \(2583\) & \(2768\)  & \(2947\)   & \(3101\)   \\ \hline
      \(n_{gen}\-[ot/min]\)       & \(50\)    & \(167\)    & \(303\)  & \(518\)   & \(1691\)   & \(1919\)   \\ \hline
      \(V_{air}\-[ms^{-1}]\)      & \(3.2\)   & \(3.7\)	   & \(4.1\)  & \(4.7\)   & \(4.9\)    & \(5.2\)    \\ \hline
      \(f_{puls}\-[Hz]\)          & \(4.13\)  & \(16.949\) & \(28.9\) & \(56.18\) & \(149.25\) & \(187.27\) \\ \hline
    \end{tabular}
    \caption{\label{tabulka_mereni} První úloha}
  \end{table}
\end{minipage}
Příklady výpočtů: \\
\\
\large
\(
  p = \frac{f_{puls-i}\cdot 60}{n_{gen-i}} = \frac{187.27 \cdot 60}{1919} = 5.855 => \) generátor má šest pólů 
%\)
\normalsize \\

\subsection{Závěr}
Podle našeho měření má generátor šest pólů.
Startovací rychlost větru určuji na \(4.8\-[ms^{-1}]\), protože kolem této hodnoty prudce rostou otáčky generátoru.



\end{document}