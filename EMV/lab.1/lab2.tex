\documentclass{article}
\usepackage{graphicx}
\usepackage{wrapfig}
\usepackage{filecontents}
\usepackage{siunitx}
\usepackage[table]{xcolor}
\usepackage{float}
\usepackage{hyperref}

\usepackage{color} % balíček pro obarvování textů
\usepackage{xcolor}  % zapne možnost používání barev, mj. pro \definecolor
\usepackage{pgfplots} % http://www.chiark.greenend.org.uk/doc/texlive-doc/latex/pgfplots/pgfplots.pdf

\usepackage{blindtext}

\ifnum 0\ifxetex 1\fi\ifluatex 1\fi=0 % if pdftex
  \usepackage[T1]{fontenc}
  \usepackage[utf8]{inputenc}
\else % if luatex or xelatex
  \ifxetex
    \usepackage{mathspec}
  \else
    \usepackage{fontspec}
  \fi
  \defaultfontfeatures{Ligatures=TeX,Scale=MatchLowercase}
\fi
\usepackage[total={175mm,230mm}, top=23mm, left=20mm, includefoot]{geometry}
\hypersetup{
    colorlinks,
    linkcolor={blue!50!black},
    citecolor={green!50!black},
    urlcolor={blue!80!black}
}
% \definecolor{fialova}{RGB}{ 255, 000, 255}
\definecolor{color-si1}{RGB}{ 255, 000, 000}
\definecolor{color-si2}{RGB}{ 251, 130, 032}

\definecolor{color-ge1}{RGB}{ 000, 255, 000}
\definecolor{color-ge2}{RGB}{ 032, 251, 160}

\definecolor{color-inp1}{RGB}{ 000, 000, 255}
\definecolor{color-inp2}{RGB}{ 160, 032, 251}

\definecolor{color-geas1}{RGB}{ 225, 225, 000}
\definecolor{color-geas2}{RGB}{ 225, 225, 100}

\definecolor{sedak}{RGB}{ 100, 100, 100}


\newcommand \obr[1]
{ obr.~\ref{#1}}

\newcommand \tab[1]
{ tab.~ß\ref{#1}}


% \documentclass{article}
% \usepackage{graphicx}
% \usepackage{wrapfig}
% \usepackage{filecontents}
% \usepackage{siunitx}
% \usepackage[table]{xcolor}
% \usepackage{float}
% \usepackage{hyperref}
% \usepackage{color} % balíček pro obarvování textů
% \usepackage{xcolor}  % zapne možnost používání barev, mj. pro \definecolor
% \hypersetup{
%     colorlinks,
%     linkcolor={red!50!black},
%     citecolor={green!50!black},
%     urlcolor={blue!80!black}
% }
% \definecolor{orange}{RGB}{ 251, 114, 032}
% \definecolor{fialova}{RGB}{ 255, 000, 255} 

% \usepackage{lmodern}
% \usepackage{amssymb,amsmath}
% \usepackage{ifxetex,ifluatex}
% \usepackage{fixltx2e} % provides \textsubscript
% \ifnum 0\ifxetex 1\fi\ifluatex 1\fi=0 % if pdftex
%   \usepackage[T1]{fontenc}
%   \usepackage[utf8]{inputenc}
% \else % if luatex or xelatex
%   \ifxetex
%     \usepackage{mathspec}
%   \else
%     \usepackage{fontspec}
%   \fi
%   \defaultfontfeatures{Ligatures=TeX,Scale=MatchLowercase}
% \fi






% \usepackage{subfiles} % Best loaded last in the preamble

% \usepackage{bookmark}
% \usepackage{tikz}
% \usetikzlibrary{patterns}

% \usepgfplotslibrary{polar}
% \usepgfplotslibrary{external}
% \usepgfplotslibrary{fillbetween}



\begin{document}
\input{Titulni_strana_protokolu.tex}

\section*{Zadání}
S pomocí grafické metody stanovte pro vybrané izolační materiály bod v jejich H-N diagramu
pro který platí \((\omega \tau) = 1\), a na základě parametrů tohoto bodu odsimulujte pomocí programu
Havriliak-Negami.bat hodnoty distribučních parametrů \(\alpha\) a \(\beta\).
Vyhodnoťte, jak se liší simulovaný průběh od naměřeného a pokuste se na H-N diagramu najít
bod pro který dojde k co nejlepší shodě mezi simulovaným a naměřeným průběhem. Zhodnoťte
jak se od sebe liší graficky odečtený bod \((\omega \tau) = 1\) a bod nejlepší shody.
Na základě hodnot \(\alpha\) a \(\beta\) pro bod nejlepší shody vypočtěte kmitočtovou závislost dielektrické
konstanty \(\varepsilon`\) a ztrátového čísla \(\varepsilon``\) na kmitočtu (minimálně pro 20 hodnot) a sestrojte H-N
diagram.

\section*{Teoretický úvod}
Za předpokladu jedné relaxační doby platí pro komplexní permitivitu dielektrika vztah \\
\\
\Large
\(
    \varepsilon^* = \varepsilon_\infty + \frac{\varepsilon_s - \varepsilon_\infty}{1 + j \omega \tau} 
\)
\\
\\
\normalsize
Kde \(\varepsilon_s\) je statická relativní permitivita (při \(\omega = 0\)) \(\varepsilon_\infty\) je optická relativní permitivita (při \(\omega \rightarrow \infty \)) \(\tau\) je relaxační doba.
\begin{figure}[H]
    \centering
    \includegraphics[width=0.7\textwidth]{obrazky/teorie1.png}
\end{figure}
Platí však jen pro některé dielektrika, a proto se postupně do vztahu přidaly další dva parametry (\(\alpha\) a \(\beta\))\\
\\
\Large
\(
    \varepsilon^* = \varepsilon_\infty + \frac{\varepsilon_s - \varepsilon_\infty}{\left[1 + (j \omega \tau)^{1-\alpha}\right]^{\beta}} 
\)
\\
\\
\normalsize
Kde \(\alpha\) je distribuční parametr, který posouvá střed části kružnice ve vertikálním směru (pod osu) a \(\beta\) je činitel, který křivku deformuje, př. viz simulace.

Na Havriliakově - Negamiho diagramu existuje bod, pro který platí \(\omega \tau = 1\) a lze ho určit graficky jak je znázorněno níže.
\begin{figure}[H]
    \centering
    \includegraphics[width=0.6\textwidth]{obrazky/teorie2.png}
\end{figure}
Při znalosti kmitočtu pak lze určit \(\tau\) a následně i koeficienty \(\alpha\) a \(\beta\).


\newpage
\subsection*{Měření}
% \begin{tabular}{|c|c|c|}
%     \hline
%     Teplota \(25.1\-[^\circ C]\) & vzdušná vlhkost \(37.2\-[\%RH]\) & atm. tlak \(p = 102.6\-[hPa]\) \\ \hline
% \end{tabular}

\begin{figure}[H]
    % \hfill
    \begin{minipage}[t]{\textwidth}
        \centering
        \begin{tikzpicture}
            \begin{axis}[
                width=\textwidth, 
                height=0.6\textwidth,
                title={chloprenový kaučuk},
                xlabel={\(T\-[K]\)}, 
                ylabel={\(W\-[eV]\)},
                xmin=2.5, xmax=6.2,
                ymin=0.0, ymax=1,
                xtick={0,0.2,...,6.5},
                ytick={0,0.1,...,1.5},
                legend pos=north west,
                ]
                \addplot[
                color=blue,
                ]
                coordinates {
                    (6.10, 0.00)
                    (5.85, 0.26)
                    (5.66, 0.42)
                    (5.51, 0.50)
                    (5.33, 0.57)
                    (5.24, 0.59)
                    (5.10, 0.63)
                    (4.94, 0.65)
                    (4.75, 0.66)
                    (4.52, 0.65)
                    (4.22, 0.61)
                    (3.85, 0.53)
                    (3.65, 0.46)
                    (3.46, 0.40)
                    (3.36, 0.36)
                    (3.12, 0.27)
                    (2.91, 0.20)
                    (2.80, 0.15)
                    (2.63, 0.08)
                    (2.53, 0.04)
                };
            \addplot[
                    dotted,
                    thick,
                    color=black,
                ]
                coordinates {
                    (2.43 , 0)
                    (2.53 , 0.04)
                    (4.8  , 1)
                };
            \addplot[
                dotted,
                thick,
                color=black,
                ]
                coordinates {
                    (2.4  , 0)
                    (7    , 0.75)
                };
            \addplot[
                    dotted,
                    thick,
                    color=black,
                    mark=o
                ]
                coordinates {
                    (5.496, 0)
                    (5.496, 0.505)
                };
            \end{axis}
        \end{tikzpicture}
    \end{minipage}
    
    \begin{minipage}[t]{\textwidth}
        \centering
        \includegraphics[width=0.8\textwidth]{obrazky/kaucuk_vypoctene.png}
    \end{minipage}
\end{figure}

\begin{figure}
    \begin{minipage}[t]{\textwidth}
        \centering
        \begin{tikzpicture}
            \begin{semilogxaxis}[
                width=\textwidth, 
                height=0.6\textwidth,
                title={Závislost složek komplexní permitivity na kmitočtu},
                xlabel={\(f\-[Hz]\)}, 
                ylabel={\(\varepsilon\-[?]\)},
                xmin=0.1, xmax=10^15,
                % ymin=0.0, ymax=1,
                % xtick={0,0.2,...,6.5},
                % ytick={0,0.1,...,1.5},
                legend pos=north east,
                ]
                \addplot[
                    color=red,
                    thick,
                    domain=0.1:10^15,
                    samples=500,
                    %   color=red,
                    %   mark=x,
                    ]
                    {2.53+(((((1.94*10^(-3)*x*2*3.14159)^(1-0.35))*sin(0.35*3.14159/2)^2)+(1+((1.94*10^(-3)*x*2*3.14159)^(1-0.35))*sin(0.35*3.14159/2))^2)^(-0.364/2))*(6.1-2.53)*cos(0.364*atan((((1.94*10^(-3)*x*2*3.14159)^(1-0.35))*cos(0.35*3.14159/2))/(((1+1.94*10^(-3)*x*2*3.14159)^(1-0.35))*sin(0.35*3.14159/2))))};
                \addlegendentry{\scriptsize \(\varepsilon'\)}
                \addplot[
                    color=green,
                    thick,
                    domain=0.1:10^15,
                    samples=500,
                    %   color=red,
                    %   mark=x,
                    ]
                    {(((((1.94*10^(-3)*x)^(1-0.35))*sin(0.35*3.14159/2)^2)+(1+((1.94*10^(-3)*x)^(1-0.35))*sin(0.35*3.14159/2))^2)^(-0.364/2))*(6.1-2.53)*sin(0.364*atan((((1.94*10^(-3)*x)^(1-0.35))*cos(0.35*3.14159/2))/(((1+1.94*10^(-3)*x)^(1-0.35))*sin(0.35*3.14159/2))))};
                \addlegendentry{\scriptsize \(\varepsilon''\)}
            \end{semilogxaxis}
        \end{tikzpicture}
    \end{minipage}
\end{figure}


\subsection*{Závěr}
Porovnáním simulace a grafické metody, ukázalo nepříliš dobrou shodu.
To je však pravděpodobně zapříčiněno především špatným odhadem frekvence v bodě \(\omega \tau = 1\).

\end{document}